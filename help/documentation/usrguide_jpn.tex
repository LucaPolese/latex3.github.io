% \iffalse meta-comment
%
% Copyright 1993-2015
% The LaTeX3 Project and any individual authors listed elsewhere
% in this file. 
% 
% This file is part of the LaTeX base system.
% -------------------------------------------
% 
% It may be distributed and/or modified under the
% conditions of the LaTeX Project Public License, either version 1.3c
% of this license or (at your option) any later version.
% The latest version of this license is in
%    http://www.latex-project.org/lppl.txt
% and version 1.3c or later is part of all distributions of LaTeX 
% version 2005/12/01 or later.
% 
% This file has the LPPL maintenance status "maintained".
% 
% The list of all files belonging to the LaTeX base distribution is
% given in the file `manifest.txt'. See also `legal.txt' for additional
% information.
% 
% The list of derived (unpacked) files belonging to the distribution 
% and covered by LPPL is defined by the unpacking scripts (with 
% extension .ins) which are part of the distribution.
% 
% \fi
% Filename: usrguide.tex
%%
% ご覧の日本語訳は、もともとは私的目的で,上記の機関とは無関係につくりました.
% 内容の正確性については原文を確認してください.
%These Japanese translations were originally created for private purposes, regardless of the above organization. 
%Please check the original about the accuracy of the contents.
%%%

\NeedsTeXFormat{LaTeX2e}[1995/12/01]

\documentclass{ltxguide}[2001/05/28]
%% for Japanese; lualaex-ja
%% \usepackage{luatexja}
\setlength{\parindent}{0.98em}% 1字下にします
%%
%\title{\LaTeXe~for authors}
\title{著者のための\LaTeXe}

\author{\copyright~Copyright 1995--2015, \LaTeX3 Project Team.\\
   All rights reserved.}

\date{30 March 2015;日本語訳は2018-09-04です.}


\begin{document}

\maketitle

\tableofcontents

%\section{Introduction}
\section{イントロダクション}

%Welcome to \LaTeXe, the new standard version of the \LaTeX{} Document
%Preparation System.
ドキュメント処理システム\LaTeX{}の新しい標準である\LaTeXe{}にようこそ.

%This document describes how to take advantage of the new features of
%\LaTeX, and how to process your old \LaTeX{} documents with
%\LaTeXe. However, this document is only a brief introduction to the
%new facilities and is intended for authors who are already familiar
%with the old version of \LaTeX{}.  It is \emph{not} a reference manual
%for \LaTeXe{} nor is it a complete introduction to \LaTeX.
このドキュメントでは,新しい\LaTeX{}の機能の使い方と,
あなたが書いた古い\LaTeX{}を\LaTeXe{}で再利用する方法を説明します.
このドキュメントは新しい機能についての簡単な入門であり,
さらに読者は著者として古い\LaTeX{}を使ったことがあることを想定しています.
このドキュメントは\LaTeXe{}のリファレンスマニュアルでも\emph{なければ},
完全な紹介を与えているわけでも\emph{ありません}.

%It is somewhat of an historical document now, since \LaTeXe{} came into
%existence in 1994.
これはいまでは歴史期的なドキュメントです.なぜなら\LaTeXe{}は1994年から存在しています.

%\subsection[\LaTeXe---The new \LaTeX~release]
%  {\LaTeXe---The new \LaTeX~release\\ (well, for more than 10 years now)}
\subsection[\LaTeXe--- 新しい\LaTeX{}の登場]
  {\LaTeXe---新しい\LaTeX{}の登場\\ (とはいうものの,今から10年以上も昔ですね.)}

%The previous version of \LaTeX{} was known as \LaTeX~2.09.  Over the
%years many extensions have been developed for \LaTeX.  This is, of
%course, a sure sign of its continuing popularity but it has had one
%unfortunate result: incompatible \LaTeX{} formats came into use at
%different sites.  This included `standard \LaTeX~2.09', \LaTeX{} built
%with the \emph{New Font Selection Scheme}~(\NFSS), \SLiTeX, \AmSLaTeX,
%and so on.  Thus, to process documents from various places, a site
%maintainer was forced to keep multiple versions of the \LaTeX{}
%program.  In addition, when looking at a source file it was not always
%clear for which format the document was written.
以前のバージョンの\LaTeX{}は\LaTeX{}~2.09として知られていました.
長年にわたり,\LaTeX{}用に多くの拡張が開発されました.
これはもちろん継続的な人気のおかげですが,不幸な結果ももたらしました.
互換性のない\LaTeX{}フォーマットが、いろいろな場所で使用されたのです.
それらは,`標準\LaTeX{}~2.09',\emph{New Font Selection Scheme}(\NFSS)で強化された\LaTeX{},\SLiTeX{},\AmSLaTeX{}などです.
したがって,さまざまな場所で作成されたドキュメントを処理するために,
システム管理者は複数のバージョンの\LaTeX{}プログラムを維持管理しなければなりませんでした.
さらに,ソースファイルをみても,そのドキュメントが,どのフォーマットの元で書かれたのかが常に明確であるとは限りませんでした.

%To put an end to this unsatisfactory situation, \LaTeXe{} has been
%produced; it brings all such extensions back under a single format and
%thus prevents the proliferation of mutually incompatible dialects of
%\LaTeX~2.09.  With \LaTeXe{} the `new font selection scheme' is
%standard and, for example, \textsf{amsmath} (formerly the \AmSLaTeX{}
%format) or \textsf{slides} (formerly the \SLiTeX{} format) are simply
%extensions, which may be loaded by documents using the same base format.
この不満足な状況に終止符を打つために,\LaTeXe{}が作られました;
\LaTeXe{}は,そのようなすべての拡張を単一フォーマットにまとめあげて,\LaTeXe{}~2.09との
互換性のない方言が広まるのを防ぎます.
\LaTeXe{}では,`新しいフォント選択スキーム(new font selection scheme)'が標準となり,
たとえば,\textsf {amsmath}(以前は\AmSLaTeX{}フォーマット)や
\textsf{slides}(以前は\SLiTeX{}フォーマット)は単なる拡張機能となりました.
これらは同一の基本フォーマットを使用したドキュメントによって読み込まれます.

%The introduction of a new release also made it possible to add a small
%number of often-requested features and to make the task of writing
%packages and classes simpler.
新しいリリースの導入により,頻繁に要求されるいくつかの機能を追加すること,
パッケージやクラスを書く作業をシンプルにします.

%\subsection{\LaTeX3---The long-term future of \LaTeX}
%\label{Sec:ltx3}
\subsection{\LaTeX3--- \protect{\LaTeX{}}を長い将来まで使うために}
\label{Sec:ltx3}

%\LaTeXe{} is the consolidation step in a comprehensive
%reimplementation of the \LaTeX{} system.  The next major release of
%\LaTeX{} will be \LaTeX3, which will include a radical overhaul of the
%document designers' and package writers' interface to \LaTeX.
\LaTeXe{}は,\LaTeXe{}システムの包括的な再実装における統合ステップです.
\LaTeXe{}の次の大きなリリースは,\LaTeX3{}になります.
これには,ドキュメントデザイナーとパッケージ作者のための,
\LaTeXe{}へのインターフェースの抜本的な見直しが含まれます.

%\LaTeX3 is a long-term research project but, until it is completed,
%the project team are committed to the active maintenance of \LaTeXe{}.
%Thus the experience gained from the production and maintenance of
%\LaTeXe{} will be a major influence on the design of \LaTeX3.
%A brief description of the project can be found in the document
%|ltx3info.tex|.
\LaTeX3{}は長期的な研究プロジェクトですが,
プロジェクトチームは完了するまでは,\LaTeXe{}に対して積極的なメンテナンスを行います.
したがって,\LaTeXe{}の提供とメンテナンスから得られた経験は,\LaTeX3{}の設計に
大きな影響を与えます.
プロジェクトの簡単な説明は,|ltx3info.tex|にあります.

%If you would like to support the project then you are welcome to send
%donations to the \LaTeX3 Project Fund; this has been set up to help
%the research team by financing various expenses associated with this
%voluntary work of maintaining the current \LaTeX{} and developing
%\LaTeX3.
このプロジェクトをサポートしてくださるなら,\LaTeX3{}プロジェクト基金(\LaTeX3 Project Fund)に寄付をお寄せください;
これは現在の\LaTeX{}を維持し,\LaTeX3{}を開発するために必要な
さまざまな経費について,研究チームを経済的に支援するために設立されました.

%The fund is administered by The \TeX{} Users Group and by various
%local user groups.  Information about making donations and joining
%these groups is available from:
%\begin{quote}\small\label{addrs}
%   \texttt{http://www.tug.org/lugs.html}
%\end{quote}
基金はThe \TeX{} Users Groupや,それぞれの地域のユーザグループによって管理されます.
寄付をされる場合や,ユーザグループに参加するための情報は次のところにあります:
\begin{quote}\small\label{addrs}
   \texttt{http://www.tug.org/lugs.html}
\end{quote}


%The \LaTeX3{} project has its home page
%on the World Wide Web at:
%\begin{verbatim}
%  http://www.latex-project.org/
%\end{verbatim}
%This page describes \LaTeX{} and the \LaTeX3 project, and contains
%pointers to other \LaTeX{} resources, such as the user guides, the
%\TeX{} Frequently Asked Questions, and the \LaTeX{} bugs database.
\LaTeX3{}プロジェクトのホームページは
\begin{verbatim}
  http://www.latex-project.org/
\end{verbatim}
です.ここには\LaTeX{}と\LaTeX3{}についての紹介があります.
そして,\LaTeX{}についての他の情報源の案内があります.
たとえば,ユーザガイド,\TeX{}で頻繁に尋ねられる質問(\TeX{} Frequently Asked Questions),
そして\LaTeX{}のデータベースがあります.

%Older articles covering aspects of the \LaTeX3 project are also
%available for anonymous ftp from the Comprehensive \TeX{} Archive, in
%the directory:
%\begin{verbatim}
%  ctan:info/ltx3pub
%\end{verbatim}
%The file |ltx3pub.bib| in that directory contains an abstract of each
%of the files.
古いドキュメントですが\LaTeX3{}についてのドキュメントはComprehensive \TeX{} Archive
の元で,ディレクトリ
\begin{verbatim}
  ctan:info/ltx3pub
\end{verbatim}
から,匿名ftpで公開されています.
このディレクトリにある|ltx3pub.bib|には,それぞれのファイルの概略が書かれています.


%\subsection{Overview}
\subsection{大まかな内容}

%This document contains an overview of the new structure and features
%of \LaTeX.  It is \emph{not} a self-contained document, as it contains
%only the features of \LaTeX{} which have changed since version 2.09.
%You should read this document in conjunction with an introduction to
%\LaTeX{}.
このドキュメントには,\LaTeX{}の新しい構造と機能の概要が含まれています.
これはバージョン2.09以降に変更された\LaTeX{}の機能だけを含んでいるので,
自己完結型のドキュメントでは\emph{ありません}.
読者は\LaTeX{}の紹介と合わせてこのドキュメントを読むべきです.

\begin{description}

%\item[Section~\ref{Sec:class+packages}]
%   contains an overview of the new structure of
%   \LaTeX{} documents.  It describes how classes and packages work and
%   how class and package options can be used.  It lists the standard
%   packages and classes which come with \LaTeX.
\item[セクション~\ref{Sec:class+packages}]
ここでは\LaTeX{}ドキュメントの新しい構造についての大まかな説明です.
クラスとパッケージがどのように働くのか,そしてそれらのオプションの使い方を説明しています.
 \LaTeX{}と一緒に提供されている標準パッケージとクラスの一覧を示しています.

%\item[Section~\ref{Sec:commands}] describes the new commands available
%   to authors in \LaTeXe.
\item[セクション~\ref{Sec:commands}] 
ここでは\LaTeXe{}の著者のための新しいコマンドを説明しています.

%\item[Section~\ref{Sec:209}] shows how to process old \LaTeX{}
%   documents with \LaTeXe.
\item[セクション~\ref{Sec:209}]
ここでは,\LaTeXe{}で古い\LaTeX{}を処理する方法を示しています.

%\item[Section~\ref{Sec:problems}] contains advice on dealing with
%   problems you may encounter in running \LaTeXe.
%   It lists some error messages which are new in \LaTeXe{} and
%   it describes some of the more common problems and how to cure them,
%   or where to find further information.
\item[セクション~\ref{Sec:problems}] 
ここでは\LaTeXe{}を使っていると目にする問題についての解決法のアドバイスを説明しています.
\LaTeXe{}ユーザにとっては,新しいいくつかのエラーメッセージと,
よく現れる問題と,それへの対処方法,あるいは,より詳しい情報が期にあるかを示しています.
   
\end{description}

%\subsection{Further information}
\subsection{追加の情報}

%For a general introduction to \LaTeX, including the new features of
%\LaTeXe, you should read \emph{\LaTeXbook}
%by Leslie Lamport~\cite{A-W:LLa94}.
新しい機能を含んだ\LaTeX{}である\LaTeXe{}の一般的な入門として,
必ずLeslie Lamport\cite{A-W:LLa94}の著書\emph{\LaTeXbook}を読むべきです.

%A more detailed description of the new features of \LaTeX, including an
%overview of more than 200 packages and nearly 1000 ready to run examples, is
%to be found in \emph{\LaTeXcomp second edition} by Frank Mittelbach and
%Michel Goossens~\cite{A-W:MG2004}.
\LaTeX{}について,さらに詳しいことは,
200のパッケージと,すぐに実行できる
1000の例とともに
Frank MittelbachとMichel Goossens\cite{A-W:MG2004}の著書
\emph{\LaTeXcomp\ second edition}に書かれています.

%Packages and programs for producing and manipulating graphics are
%discussed at length in \emph{\LaTeXGcomp} by Michel Goossens,
%Sebastian Rahtz and Frank Mittelbach~\cite{A-W:GRM97}.
グラフィックスの生成と操作についてのパッケージとプログラムについては
Michel GoossensとSebastian RahtzとFrank Mittelbachの著書\cite{A-W:GRM97}\emph{\LaTeXGcomp}
に詳しく説明されています

%Solutions for publishing with \LaTeX{} on the World Wide Web are given
%in \emph{\LaTeXWcomp} by Michel Goossens and Sebastian
%Rahtz~\cite{A-W:GR99}.
\LaTeX{}で作成したドキュメントをWorld Wide Webで公開することについては
Michel GoossensとSebastian Rahtz\cite{A-W:GR99}の著書
\emph{\LaTeXWcomp}で説明されています.

%For more information about the many new \LaTeX{} packages you should
%read the package documentation, which should be available from the
%same source as your copy of \LaTeX.
新しい\LaTeX{}のパッケージについては,
\LaTeX{}と一緒に配布されている
それぞれのパッケージに付属するドキュメントを読んでください.

%There are a number of documentation files which accompany every copy
%of \LaTeX.  A copy of \emph{\LaTeX{} News} will come out with each
%six-monthly release of \LaTeX; it will be found in the files
%|ltnews*.tex|.  The class- and package-writer's guide \emph{\clsguide}
%describes the new \LaTeX{} features for writers of document classes
%and packages; it is in |clsguide.tex|.  The guide \emph{\fntguide}
%describes the \LaTeX{} font selection scheme for class- and
%package-writers; it is in |fntguide.tex|. Support for Cyrillic languages
%in \LaTeX{} is described in \emph{\cyrguide}.
\LaTeX{}は多数の解説ドキュメントと一緒に配布されます.\LaTeX{}は6ヶ月ごとにリリースされ,それには\emph{\LaTeX{} News}が含まれます;内容はファイル|ltnews*.tex|にあります.クラスとパッケージ作成者ためのガイドが\emph{\clsguide}ですが,これには新しい\LaTeX{}機能にもとづいたドキュメントクラスとパッケージの作者のための説明です;ファイル|clsguide.tex|にあります.ガイド\emph{\fntguide}は,クラスライターとパッケージライターのための\LaTeX{}フォント選択スキームを説明しています;|fntguide.tex|にあります.\LaTeX{}でのキリル語のサポートは,\emph{\cyrguide}に記述されています.

%The documented source code (from the files used to produce the kernel
%format via |latex.ltx|) is now available as
%\emph{The \LaTeXe\ Sources}.  ]
%This very large document also includes an index of
%\LaTeX{} commands.  It can be typeset from the \LaTeX{} file
%|source2e.tex| in the |base| directory, using the source files and
%the class file |ltxdoc.cls| from this directory.
ドキュメントのソースコード(|latex.ltx|からカーネル形式を生成するために使用されるファイルから生成されます)は,\emph{The LaTeXe\ Sources}にあります.この非常に大きなドキュメントには,\LaTeX{}コマンドのインデックスも含まれています.これは,|base|ディレクトリで\LaTeX{}ファイル|source2e.tex|を,ここにあるソースファイルとクラスファイル|ltxdoc.cls|を使用してタイプセットすることで得られます.

%For more information about \TeX{} and \LaTeX{}, please contact your
%local \TeX{} Users Group, or the international \TeX{} Users Group (see
%page \pageref{addrs}).
\TeX{}と\LaTeX{}についてもっと詳しく知りた人は,お近くの\TeX{}ユーザーズグループ,または国際的な\TeX{} Usersグループ(\pageref {addrs}参照)にお問い合わせください.

%\section{Classes and packages}
\section{クラストパッケージ}
\label{Sec:class+packages}

%This section describes the new structure of \LaTeX{} documents and the
%new types of file: \emph{classes} and \emph{packages}.
このセクションでは,\LaTeX{}ドキュメントの新しい構造と,\emph{classes}と\emph{packages}という新しいタイプのファイルについて説明します.

%\subsection{What are classes and packages?}
\subsection{クラスとパッケージと何か}

%The main difference between \LaTeX~2.09 and \LaTeXe{} is in the
%commands before |\begin{document}|.
\LaTeX~2.09と\LaTeXe{}の主な違いは|\begin{document}|の前に置くコマンドです.

%In \LaTeX~2.09, documents had \emph{styles},
%such as \textsf{article} or \textsf{book}, and \emph{options},
%such as \textsf{twoside} or \textsf{epsfig}.
%These were indicated by the |\documentstyle| command:
\LaTeX~2.09ではドキュメントの種類を決めるために\textsf{article}あるいは\textsf{book}という\emph{スタイル}が使われていました.
そして \textsf{twoside}あるいは\textsf{epsfig}のような\emph{オプション}を指定していました.
これらは|\documentstyle|によって指定されるようになりました.
\begin{quote}
   |\documentstyle|\oarg{options}\arg{style}
\end{quote}
%For example, to specify a two-sided article with encapsulated
%PostScript figures, you said:
たとえば,二段組み(two-sided)の論文(article)で,ポストスクリプトの図が使われているなら:
\begin{verbatim}
   \documentstyle[twoside,epsfig]{article}
\end{verbatim}
のようにします.
%However, there were two different types of document style option:
%\emph{built-in options} such as |twoside|; and \emph{packages} such as
%|epsfig.sty|.  These were very different, since any \LaTeX{} document
%style could use the \textsf{epsfig} package but only document styles
%which declared the \textsf{twoside} option could use that option.
しかしながら,2種類の異なるドキュメントのスタイルオプションがあります:
|twoside|のような\emph{組み込み}のオプションと
|epsfig.sty|のような\emph{パッケージ}を要求雨するものです.
これらは全く異なる目的をもっています.
どのような\LaTeX{}ドキュメントスタイルでも\textsf{epsfig}パッケージを使うことがありますが,
\textsf{二段組}と宣言されたドキュメントは,そのドキュメントスタイルとなります.

%To avoid this confusion, \LaTeXe{} differentiates between built-in
%options and packages.  These are given by the new |\documentclass| and
%|\usepackage| commands:
このような混乱を避けるために\LaTeXe{}では,組み込みのオプションとパッケージとを区別します.
新しく|\documentclass|
と|\usepackage|コマンドを用い
\begin{quote}
   |\documentclass|\oarg{options}\arg{class} \\
   |\usepackage|\oarg{options}\arg{packages}
\end{quote}
のようにします.
%For example, to specify a two-sided article with encapsulated
%PostScript figures, you now write:
たとえば,二段組の論文でポストスクリプト(eps)の図を用いるなら
\begin{verbatim}
   \documentclass[twoside]{article}
   \usepackage{epsfig}
\end{verbatim}
のようにします.
%You can load more than one package with a single |\usepackage|
%command; for example, rather than writing:
1行の|\usepackage|コマンドで複数のパッケージの読み込みを指定できます.
つまり
\begin{verbatim}
   \usepackage{epsfig}
   \usepackage{multicol}
\end{verbatim}
%you can specify:
のように書く代わりに,まとめて
\begin{verbatim}
   \usepackage{epsfig,multicol}
\end{verbatim}
のようにすることもできます.
%Note that \LaTeXe{} still understands the \LaTeX~2.09 |\documentstyle|
%command.  This command causes \LaTeXe{} to enter \emph{\LaTeX~2.09
%compatibility mode}, which is described in Section~\ref{Sec:209}.
新しい\LaTeXe{}は古い\LaTeX~2.09の|\documentstyle|が処理できるようになっています.
このとき\LaTeXe{}は\emph{\LaTeX~2.09互換性モード}にはいります.
これについては\ref{Sec:209}節で説明しています.

%You should not, however, use the |\documentstyle| command for new
%documents because this compatibility mode is very slow and the new
%features of \LaTeXe{} are not available in this mode.
しかし,新しく描くドキュメントに|\documentstyle|コマンドを使ってはいけません.
互換性モードの処理は遅く,\LaTeXe{}の新しい特徴を使うことができません.

%To help differentiate between classes and packages, document classes
%now end with |.cls| rather than |.sty|.  Packages still end with
%|.sty|, since most \LaTeX~2.09 packages work well with \LaTeXe.
クラスとパッケージとの違いですが,ドキュメントクラスは拡張子が|.sty|ではなくて|.cls|
であることです.
パッケージの拡張子は|.sty|のままで,変更されていません.
したがって,ほとんどの\LaTeX~2.09のパッケージは,そのまま\LaTeXe{}でも
動作します.

%\subsection{Class and package options}
\subsection{クラスとパッケージのオプション}

%In \LaTeX~2.09, only document styles could have options such as
%|twoside| or |draft|.  In \LaTeXe{}, both classes and packages are
%allowed to have options.  For example, to specify a two-sided article
%with graphics using the |dvips| driver, you write:
\LaTeX~2.09{}では,ドキュメントスタイルだけが|twoside|や|draft|のようなオプションをもつことができます.
\LaTeXe{}では,クラスもパッケージもオプションをもつことができます.
たとえば,二段組でグラフィックスのある論文の場合は,|dvips|ドライバに対して
\begin{verbatim}
   \documentclass[twoside]{article}
   \usepackage[dvips]{graphics}
\end{verbatim}
のように指示します.

%It is possible for packages to share common options.  For example,
%you could, in addition, load the \textsf{color} package by specifying:
パッケージは共通のオプションを併用できます.
たとえば,\textsf{color}パッケージを用いるならば
\begin{verbatim}
   \documentclass[twoside]{article}
   \usepackage[dvips]{graphics}
   \usepackage[dvips]{color}
\end{verbatim}
のようにします.

%But because |\usepackage| allows more than one package to be listed,
%this can be shortened to:
|\usepackage|は複数のオプションを与えることができるので,
\begin{verbatim}
   \documentclass[twoside]{article}
   \usepackage[dvips]{graphics,color}
\end{verbatim}
のように短く書くこともできます.

%In addition, packages will also use each option given to
%the |\documentclass| command (if they know what to do with it), so you
%could also write:
さらに,パッケージは,それぞれのオプションが適切なら|\documentclass|コマンドに与えること
もできるので,
\begin{verbatim}
   \documentclass[twoside,dvips]{article}
   \usepackage{graphics,color}
\end{verbatim}
のように書くこともできます.

%Class and package options are covered in more detail in
%\emph{\LaTeXcomp} and in \emph{\clsguide}.
クラスとパッケージのオプションについて詳しいことは,
\emph{\LaTeXcomp}と\emph{\clsguide}で説明されています.

%\subsection{Standard classes}
\subsection{標準のクラス}

%The following classes are distributed with \LaTeX:
次に示すクラスは\LaTeX{}と一緒に配布されています.
\begin{description}

\item[article]  % The |article| class described in \emph{\LaTeXbook}.
論文用の|article|については\emph{\LaTeXbook}を参照してください.
\item[book]     % The |book| class described in \emph{\LaTeXbook}.
書籍用の|book|については\emph{\LaTeXbook}を参照してください.
\item[report]   % The |report| class described in \emph{\LaTeXbook}.
レポート用の|report|については\emph{\LaTeXbook}を参照してください.
\item[letter]   % The |letter| class described in \emph{\LaTeXbook}.
手紙用の|letter|については\emph{\LaTeXbook}を参照してください.
\item[slides]   % The |slides| class described in \emph{\LaTeXbook},
 %   formerly \SLiTeX.
スライド用の|slides|については\emph{\LaTeXbook}を参照してください.
以前は\SLiTeX{}と呼ばれていたものです.
\item[proc]     % A document class for proceedings, based on |article|.
 %   Formerly the |proc| package.
 これは論文集のためのもので,|article|を基にして作られました.
 以前は|proc|パッケージでした.
\item[ltxdoc]   % The document class for documenting the \LaTeX{}
%    program, based on |article|.
\LaTeX{}を説明するドキュメント作成のためのドキュメントクラスです.
|article|を基にして作られています.
\item[ltxguide] % The document class for \emph{\usrguide} and
%   \emph{\clsguide}, based on |article|.  The document you are reading
%   now uses the |ltxguide| class. The layout for this class is likely
%   to change in future releases of \LaTeX.
\emph{\usrguide}と\emph{\clsguide}のためのドキュメントクラスです.
|article|を基にして作られています.
いま読んでいるドキュメントは,この|ltxguide|を使って書かれました.
このクラスのレイアウトは将来の\LaTeX{}がリリースされた時には変更されるでしょう.
\item[ltnews]   % The document class for the \emph{\LaTeX{} News}
%   information sheet, based on |article|. The layout for this class
%   is likely to change in future releases of \LaTeX.
\emph{\LaTeX{} News}に使われているドキュメントクラスで,|article|が基になっています.
このクラスのレイアウトは将来の\LaTeX{}がリリースされた時には変更されるでしょう.
\item[minimal]
\NEWfeature{1995/12/01}
%   This class is the bare minimum (3 lines) that is needed in a
%   \LaTeX\ class file. It just sets the text width and height, and
%   defines |\normalsize|.  It is principally intended for debugging
%   and testing \LaTeX\ code in situations where you do not need to
%   load a `full' class such as |article|. If, however, you are
%   designing a completely new class that is aimed for documents with
%   structure radically different from the structure supplied by the
%   article class, then it may make sense to use this as a base and add
%   to it code implementing the required structure, rather than
%   starting from |article| and modifying the code there.
このクラスは,\LaTeX{}クラスファイルで必要とされる最低限(3行)のものです.
テキストの幅と高さを設定し,|\normalsize|を定義するだけです.
これは主に,|article|のような `完全 'クラスをロードする必要がない状況で,\LaTeX{}コードのデバッグとテストためにあります.さらに|article|クラスで提供される構造と大きく異なる構造をもつドキュメントのために新しいクラスを,|article|の影響なく設計したいときには,これを基にして必要な構造を追加すればよいでしょう.
\end{description}

%\subsection{Standard packages}
\subsection{標準パッケージ}
\label{Sec:st-pack}

%The following packages are distributed with \LaTeX:
次のパッケージは\LaTeX{}と一緒に配布されています.
\begin{description}
\item[alltt]
\NEWfeature{1994/12/01}
%   This package provides the |alltt| environment, which is like
%   the |verbatim| environment except that |\|, |{|, and |}|
%   have their usual meanings.  It is described in |alltt.dtx| and
%   \emph{\LaTeXbook}.
このパッケージは|alltt|環境を与えます.
これは|verbatim|環境と似ていますが,|\|,|{|と|}|は,本来の意味をもちます.
詳しくは|alltt.dtx|と\emph{\LaTeXbook}を参照してください.
\item[doc] % This is the basic package for typesetting the documentation
%   of \LaTeX{} programs.  It is described in |doc.dtx| and in
%   \emph{\LaTeXcomp}.
これは\LaTeX{}プログラムのドキュメントをタイプセットするための基本パッケージです.
詳しくは|doc.dtx|と\emph{\LaTeXbook}を参照してください.
\item[exscale]  % This provides scaled versions of the math extension
%   font.  It is described in |exscale.dtx| and \emph{\LaTeXcomp}.
これは数式フォントを拡大するためにあります.
詳しくは|exscale.dtx|と\emph{\LaTeXbook}を参照してください.
 \item[fontenc] % This is used to specify which font encoding \LaTeX{}
%   should use.  It is described in |ltoutenc.dtx|.
特定のフォントエンコーディングを\LaTeX{}で使う時に使われます.
詳しくは|ltoutenc.dtx|を参照してください.
\item[graphpap]
\NEWfeature{1994/12/01}
%   This package defines the |\graphpaper| command; this
%   can be used in a |picture| environment.
このパッケージは|\graphpaper|を定義しています.
|picture|環境で使われます.
\item[ifthen]  %  Provides commands of the form `if\dots then do\dots
%   otherwise do\dots'.
%   It is described in |ifthen.dtx| and \emph{\LaTeXcomp}.
`if\dots ならば \dots そうでなければ do\dots'という形のコマンドを与えます.
詳しくは|ifthen.dtx|と\emph{\LaTeXcomp}を参照してください.
\item[inputenc]
\NEWfeature{1994/12/01}
%   This is used to specify which input encoding \LaTeX{} should use.
%   It is described in |inputenc.dtx|.
\LaTeX{}で,どの文字コードで処理するのかという指示を与えます.
詳しくは|inputenc.dtx|を参照してください.
\item[latexsym] % \LaTeXe{} no longer loads the \LaTeX{} symbol font by
%   default.  To access it, you should use the |latexsym| package.  It
%   is described in |latexsym.dtx| and in \emph{\LaTeXcomp}; see also
%   Section~\ref{Sec:problems}.
 \LaTeXe{}はデフォルトでは\LaTeX{}記号をロードしません.
 このフォントを使うのなら,|latexsym|パッケージを使います.
 詳しくは|latexsym.dtx|と\emph{\LaTeXcomp}を参照してください.
 また,\ref{Sec:problems}節にも説明があります.
 \item[makeidx] % This provides commands for producing indexes.  It is
%   described in \emph{\LaTeXbook} and in \emph{\LaTeXcomp}.
索引を生成するためのコマンドを提供します.
詳しくは\emph{\LaTeXbook}と\emph{\LaTeXcomp}を参照してください.
 \item[newlfont] % This is used to emulate the font commands of
%   \LaTeX~2.09 with the New Font Selection Scheme. It is described in
%   \emph{\LaTeXcomp}.
\LaTeX~2.09のフォントコマンドをNew Font Selection Schemeでエミュレートするために使います.
 詳しくは\emph{\LaTeXcomp}を参照してください.
 \item[oldlfont] % This is used to emulate the font commands of
%   \LaTeX~2.09.  It is described in \emph{\LaTeXcomp}.
\LaTeX~2.09のフォントコマンドをエミュレートするために使います.
 詳しくは\emph{\LaTeXcomp}を参照してください.
 \item[showidx]
%   This causes the argument of each |\index| command to
%   be printed on the page where it occurs.
%   It is described in \emph{\LaTeXbook}.
|\index|コマンドで拾われている索引項目を,それぞれのページに表示します.
 詳しくは\emph{\LaTeXbook}を参照してください.
 \item[syntonly] % This is used to process a document without
%   typesetting it.  It is described in |syntonly.dtx| and in
%   \emph{\LaTeXcomp}.
タイプセッティングをせずにドキュクメントを処理する時に使われます.
詳しくは|syntonly.dtx|と\emph{\LaTeXcomp}を参照してください.
 \item[tracefnt] % This allows you to control how much information about
%   \LaTeX's font loading is displayed.  It is described in
%   \emph{\LaTeXcomp}.
\LaTeX{}でフォントのローディングで,どれくらいの情報が使われているかを管理することができます.
詳しくは\emph{\LaTeXcomp}を参照してください.
\end{description}

%\subsection{Related software}
\subsection{補助してくれるソフトウェア}

\NEWdescription{1998/12/01}
%The following software should be available from the same distributor
%as your copy of \LaTeXe.  You should obtain at least the
%\textsf{graphics} and \textsf{tools} collections in order to have all
%the files described in \emph{\LaTeXbook}.  The |amsmath| package (part
%of \textsf{amslatex} and formerly known as |amstex|)
%and \textsf{babel} are also mentioned in the list
%of `standard packages' in section C.5.2 of that book.
次のソフトウェアは,\LaTeXe{}と一緒に同じ配布元から入手できます.
\emph{\LaTeXbook}に記載されているすべてのファイルを得るには,
少なくともグラフィックス\textsf{graphics} とツール\textsf{tools}を入手する必要があります.
|amsmath|パッケージ(\textsf{amslatex} の一部と以前は|amstex|と呼ばれていました)と\textsf{babel}については,
\emph{\LaTeXbook}のC.5.2項の `standard packages'のリストにも示されています.
\begin{description}
\item[amslatex]  % Advanced mathematical typesetting from the American
%  Mathematical Society. This includes the |amsmath| package; it
%  provides many commands for typesetting mathematical formulas of
%  higher complexity.  It is produced and supported by the American
%  Mathematical Society and it is described in \emph{\LaTeXcomp}.
アメリカ数学会のための高度な数学組版のためのパッケージです.
これには |amsmath|パッケージが含まれています.
複雑な数式を組版するための多くのコマンドを提供します.
これは,アメリカ数学会によって作成されサポートされており,\emph{\LaTeXcomp}を参照してください.
\item[babel]  % This package and related files support typesetting in
%  many languages. It is described in \emph{\LaTeXcomp}.
このパッケージおよび関連ファイルは,多言語でのタイプセットをサポートします.
詳しくは\emph{\LaTeXcomp}を参照してください.
\item[cyrillic]
\NEWfeature{1998/12/01}
%  Everything you need (except the fonts themselves) for
%  typesetting with Cyrillic fonts.
キリル文字の組版に必要なものすべて(ただしフォントは含まれていません).
\item[graphics]  % This includes the |graphics| package which
%  provides support for the inclusion and transformation of graphics,
%  including files produced by other software. Also included, is the
%  |color| package which provides support for typesetting in colour.
%  Both these packages are described in \emph{\LaTeXbook}.
これには|graphics|が含まれています.
これはグラフィックスを管理と変換,そして他のソフトウェアで作成されたグラフィックスを取り込みます.
|color|パッケージも含まれており,これはカラーでのタイプセットをサポートしています.
これらのパッケージについて詳しくは\emph{\LaTeXcomp}を参照してください.
\item[psnfss]  %  Everything you need (except the fonts themselves) for
%  typesetting with a large range of Type~1 (PostScript) fonts.
さまざまな種類のポストスクリプトのType~1フォントでタイプセットするために必要なもの(ただしフォントは含まれていません).
\item[tools]   %  Miscellaneous packages written by the \LaTeX3
%  project team.
\LaTeX3{}開発チームによって作成されたさまざまなパッケージ.
\end{description}
%These packages come with documentation and each of them is also
%described in at least one of the books \emph{\LaTeXcomp} and
%\emph{\LaTeXbook}.
これらのパッケージは,いずれもドキュメントが含まれています.
また\emph{\LaTeXcomp}と\emph{\LaTeXbook}のいずれかにも解説があります.


%\subsubsection{Tools}
\subsubsection{ツール}

%This collection of packages includes, at least, the following (some
%files may have slightly different names on certain systems):
ここに示したパッケージも,インストールされているでしょう(ただし
システムによっては,少々名前が異なるかもしれません).

\begin{description}
\item[array]
%        Extended versions of the environments |array|, |tabular|
%        and |tabular*|, with many extra features.
これらは|array|,|tabular|,|tabular*|環境に機能を加えて拡張したものです.
\item[calc]
\NEWfeature{1996/12/01}
%        Enables the use of certain algebraic notation when specifying
%        values for lengths and counters.
長さやカウンタの値を指定する時に適切な代数記号を使うためのものです.
\item[dcolumn]
%        Alignment on `decimal points' in tabular entries. Requires the
%        |array| package.
表を作成する時に`小数点'で一揃えをするためのものです.
使う時には|array|パッケージが必要です.
\item[delarray]
%        Adds `large delimiters' around arrays. Requires |array|.
配列の周りに大きな`区切り記号'が必要な時に使います.
使う時には|array|パッケージが必要です.
\item[hhline]
%        Finer control over horizontal rules in tables. Requires |array|.
表の作成の時に水平の位置の微調整をする時に使います.
使う時には|array|パッケージが必要です.
\item[longtable]
%        Multi-page tables. (Does not require |array|, but it uses the
%        extended features if both are loaded.)
複数のページにまたがる表を作成します.
(これは|array|パッケージが必要ではありませんが,両方をロードされていたら,機能が拡張されている方が使われます)
\item[tabularx]
%        Defines a |tabularx| environment that is similar to |tabular*|
%        but it modifies the column widths, rather than the inter-column
%        space, to achieve the desired table width.
|tabularx|環境を定義します.これは|tabular*|と似ていますが,
列と列の幅ではなくて,列の幅を変更し,表の幅に合わせます.
\item[afterpage]
%        Place text after the current page.
現在のページの後ろにテキストを入れます.
\item[bm]
%        Access bold math symbols.
太字の数式記号を使います.
\item[enumerate]
%        Extended version of the |enumerate| environment.
|enumerate|環境の機能拡張バージョンです.
\item[fontsmpl]
%        Package and test file for producing `font samples'.
`フォント見本'を作成するためのパッケージとテストファイルです.
\item[ftnright]
%        Place all footnotes in the right-hand column in two-column mode.
二段組みの時に脚注を右の段の下に置きます.
\item[indentfirst]
%        Indent the first paragraph of sections, etc.
節などのの最初のパラグフを字下げします.
\item[layout]
%        Show the page layout defined by the current document class.
現在使用しているドキュメントクラスでのページレイアウトを表示します.
\item[multicol]
%        Typeset text in columns, with the length of the columns
%        `balanced'.
コラムの幅を均等にして,テキストをコラムで作成します.
\item[rawfonts]
%        Preload fonts using the old internal font names of \LaTeX~2.09.
%        See Section~\ref{Sec:oldinternals}.
\LaTeX~2.09の古い内部フォント名でロードされるフォントを指定します.
\ref{Sec:oldinternals}節を参照してください.
\item[somedefs]
%       Selective handling of package options. (Used by the rawfonts
%       package.)
パッケージオプションを選択して処理します.(|rawfonts|パッケージで使用されます).
\item[showkeys]
%        Prints the `keys' used by |\label|, |\ref|, |\cite| etc.; useful
%        whilst drafting.
|\label|,|\ref|,|\cite|などで指定された`キー'を印字します.
推敲のときに使用すると便利です.
\item[theorem]
%       Flexible declaration of `theorem-like' environments.
`定理のような'環境を便利に定義します.
\item[varioref]
%       `Smart' handling of page references.
ページの参照を賢く処理をします.
\item[verbatim]
%        Flexible extension of the verbatim environment.
|verbatim|環境の便利な拡張です.
\item[xr]
%       Cross reference other `external' documents.
外部のドキュメントに対して相互参照を行います.
\item[xspace]
%       `Smart space' command that helps you to avoid the common mistake
%       of missing spaces after command names.
コマンド名の後ろにスペースが発生しないというよくある間違いを避けるための`スマートスペース'コマンドです.
\end{description}


%\section{Commands}
\section{コマンド}
\label{Sec:commands}

%This section describes the new commands available in \LaTeXe.
%They are covered in more detail in \emph{\LaTeXbook} and in
%\emph{\LaTeXcomp}.
この節では\LaTeXe{}で加わった新しいコマンドを解説します.
詳しいことは \emph{\LaTeXbook}と\emph{\LaTeXcomp}で説明されています.


%\subsection{Initial commands}
\subsection{初期化コマンド}

%Initial commands can appear only before the |\documentclass|
%line.
初期化コマンドは|\documentclass|の前に置かなければなりません.

\begin{decl}
|\begin{filecontents}| \arg{file-name} \\
  \m{file-contents} \\
|\end{filecontents}|
\end{decl}

%The |filecontents| environment is intended for bundling within a
%single document file the contents of packages, options, or other
%files.  When the document file is run through \LaTeXe{} the body of
%this environment is written verbatim (preceded by a comment line) to a
%file whose name is given as the environment's only argument.  However,
%if that file already exists then nothing happens except for an
%information message.
|filecontents|環境とは,単一のドキュメントファイルにパッケージ,オプション,またはその他のファイルの内容をまとめることを意図しています.
ドキュメントファイルが\LaTeXe{}で実行されている場合,この環境の本体は,環境の唯一の引数として名前が与えられたファイルにそのままで(verbatim)(コメント行の前に)書かれます.ただし,そのファイルがすでに存在する場合は,情報メッセージ以外は何も起こりません.

%Only normal ASCII text characters (7-bit visible text) should be
%included in a |filecontents| environment.  Anything else, such as tab
%characters, form-feeds or 8-bit characters, should not be included in a
%|filecontents| environment.
通常のASCIIテキスト文字(7ビットの可視テキスト)だけが|filecontents|環境に含まれていなければなりません.
不可視文字である,タブ文字,フォームフィード,8ビット文字などは,|filecontents|環境には含めないでください.

%Tabs and form feeds produce a warning, explaining that they are turned
%into spaces or blank lines, respectively.
%What happens to 8-bit characters depends on the \TeX{} installation and
%is in general unpredictable.
タブとフォームフィードは警告を出します.警告では,これらが,それぞれスペースや空白行に変わることを説明します.
8ビット文字は\TeX{}インストールに依存するので,何が起こるかは一般的には予測できません.

%The |filecontents| environment is used for including \LaTeX{} files.
%For other plain text files (such as Encapsulated PostScript files),
%you should use the |filecontents*| environment which does not add a
%comment line.
|filecontents|環境は\LaTeX{}ファイルをインクルードするために使用されます.
他のプレーンテキストファイル(たとえばEncapsulated PostScriptファイルなど)の場合は,
コメント行を追加しない|filecontents*|環境を使わなければなりません.

%These environments are allowed only before |\documentclass|.  This
%ensures that any packages that have been bundled in the document are
%present when needed.
これらの環境は|\documentclass|の前に置くことしか許されていません,
このように制限することで,ドキュメントにバンドルされるどのようなパッケージでも,
必要に応じて提供できるのです.

%\subsection{Preamble commands}
\subsection{プレアンブルのコマンド}
\label{Sec:pre}

%The changes to the preamble commands are intentionally designed to make
%\LaTeXe{} documents look clearly different from old documents.  The
%commands should be used only before |\begin{document}|.
プリアンブルコマンドの変更は,\LaTeXe{}ドキュメントを古いドキュメントとの違いが明確になるように意図的に設計されました.
コマンドは|\begin{document}|の前に使わなければなりません.

\begin{decl}
|\documentclass| \oarg{option-list} \arg{class-name}
   \oarg{release-date}
\end{decl}

%This command replaces the
%\LaTeX~2.09 command |\documentstyle|.
\LaTeX~2.09のコマンド|\documentstyle|を置き換えたものです.

%There must be exactly one |\documentclass| command in a document; and
%it must come after the |filecontents| environments, if any, but before
%any other commands.
ドキュメントには|\documentclass|は一つだけしか使うことができません.
そして|filecontents|環境の後ろに置かなければなりません.
また,他のコマンドよりは前に置きます.

%The \m{option-list} is a list of options, each of which may modify the
%formatting of elements which are defined in the \m{class-name} file,
%as well as those in all following |\usepackage| commands (see
%below).
\m{option-list}はオプションのリストです.
それらは\m{class-name}ファイルで定義されているフォーマット要素を変更するためのものです.
そして|\usepackage|コマンドが続きます(続いて説明します).
%
%The optional argument \m{release-date} can be used to specify the
%earliest desired release date of the class file; it should contain a
%date in the format \textsc{yyyy/mm/dd}.  If a version of the class
%older than this date is found, a warning is issued.
オプション引数\m{release-date}によって,クラスファイルの最新のリリース日を指定できます.
クラスファイルは\textsc{yyyy/mm/dd}の形式で日付を含めなければなりません.
クラスファイルのバージョンがこの日付より古い場合は,警告が出されます.

%For example, to specify a two-column article, using a version of
%|article.cls| released after June 1994, you specify:
たとえば,二段組みの論文を作成するのに,1994年6月よりも後にリリースされた
|article.cls|のバージョンを使いたい場合は
\begin{verbatim}
   \documentclass[twocolumn]{article}[1994/06/01]
\end{verbatim}
とします.

\begin{decl}
|\documentstyle| \oarg{option-list} \arg{class-name}
\end{decl}

%This command is still supported for compatibility with old files.  It
%is essentially the same as |\documentclass| except that it invokes
%\emph{\LaTeX~2.09 compatibility mode}.  It also causes any options in
%the \m{option-list} that are not processed by the class file to be
%loaded as packages after the class has been loaded. See
%Section~\ref{Sec:209} for more details on \LaTeX~2.09 compatibility
%mode.
このコマンドは降りファイルとの互換性のためにサポートされています.
本質的には|\documentclass|と同じですが,\emph{\LaTeX~2.09互換モード}
が起動します.
また,クラスファイルによって処理されなかった\m {option-list}のオプションは,
クラスのロードされた後にパッケージとしてロードされます.
詳しいことは\ref{Sec:209}にある\LaTeX~2.09{}互換モードを参照してください.

\begin{decl}
|\usepackage| \oarg{option-list} \arg{package-name} \oarg{release-date}
\end{decl}

%Any number of |\usepackage| commands is allowed. Each package file
%(as denoted by \m{package-name}) defines new elements (or modifies
%those defined in the class file loaded by the \m{class-name} argument
%of the |\documentclass| command).  A package file thus extends the
%range of documents which can be processed.
利用できる|\usepackage|の数に制限はありません.
それぞれのパッケージ(m{package-name}で示しました)は,
新しい要素(あるいは|\documentclass|コマンドの\m{class-name}引数でロードされたクラスファイル定義の変更)を定義します.
パッケージファイルは,このようにドキュメントの範囲を拡張します.

%The \m{option-list} argument can contain a list of options, each of
%which can modify the formatting of elements which are defined in this
%\m{package-name} file.
\m{option-list}引数はオプションのリストを含むことができます.
それらは\m{package-name}ファイルで定義されていたフォーマット要素を変更できます.

%As above, \m{release-date} can contain the earliest desired release
%date of the package file in the format \textsc{yyyy/mm/dd}; if an
%older version of the package is found, a warning is issued.
上述したように,\m{release-date}はパッケージファイルの日付を
\textsc{yyyy/mm/dd}というフォーマットでのみ含むことができます.
パッケージの古いバージョンが見つかった場合は,警告が出されます.

%For example, to load the |graphics| package for the |dvips| driver,
%using a version of |graphics.sty| released after June 1994, you write:
たとえば,|graphics|を|dvips|ドライバの使用するさいに,
1994年6月以降の|graphics.sty|をロードするなら,
\begin{verbatim}
   \usepackage[dvips]{graphics}[1994/06/01]
\end{verbatim}
と書きます.

%Each package is loaded only once.  If the same package is requested
%more than once, nothing happens in the second or following attempt
%unless the package has been requested with options that were not given
%in the original |\usepackage|. If such extra options are specified
%then an error message is produced. See Section~\ref{Sec:problems} how
%to resolve this problem.
それぞれのパッケーは1回だけロードします.
同じパッケージが2回以上要求されても,最初の|\usepackage|でオプションの追加の設定がなければ,2回目以降は何もしません.
追加されたオプションではエラーメッセージが表示されことがあります.
この場合,解決するためには\ref{Sec:problems} 節を参照してください.

%As well as processing the options given in the \m{option-list} of the
%|\usepackage| command, each package processes the
%\m{option-list} of the |\documentclass| command as well. This
%means that any option which should be processed by every package (to
%be precise, by every package that specifies an action for it) can be
%specified just once, in the |\documentclass| command, rather than
%being repeated for each package that needs it.
|\usepackage|コマンドの\m{option-list}で与えられたオプションを処理するだけでなく,それぞれのパッケージは|\documentclass|コマンドの\m{option-list} も処理します.
言い換えると,すべてのパッケージ(正確には動作を指定するすべてのパッケージ)で処理する必要のあるオプションは,それを必要とするパッケージごとに繰り返し指定するのではなく,|\documentclass|コマンドで一度だけ指定するということです.

\begin{decl}
|\listfiles|
\end{decl}

%If this command is placed in the preamble then a list of the files
%read in (as a result of processing the document) will be displayed
%on the terminal (and in the log file) at the end of the run. Where
%possible, a short description will also be produced.
このコマンドがプリアンブルに置かれている場合,(ドキュメントの処理の結果として)読み込まれたファイルのリストが,実行の最後に端末(およびログファイル)に表示されます.
可能であれば,短い説明も作成されます.

\NEWdescription{1995/12/01}
%\emph{Warning}: this command will list only files which were read
%using \LaTeX{} commands such as |\input|\arg{file} or
%|\include|\arg{file}.  If the file was read using the primitive \TeX{}
%syntax |\input |\emph{file} (without |{ }| braces around the file name)
%then it will not be listed; failure to use the \LaTeX{} form with the
%braces can cause more severe problems, possibly leading to overwriting
%important files, so \textbf{always put in the braces}.
\emph{警告}:このコマンドは,|\input|\arg{file}あるいは|\include|\arg{file}などの\LaTeX{} コマンドを使用して読み込まれたファイルのみを一覧表示します.
ファイルがプリミティブな\TeX{}のシンタックス|\input |\emph{file}を使用して読み込まれた場合は(ファイル名の前後に中括弧 |{ }|は付けられません),リストには表示されません.\LaTeX{}形式で中括弧を使用しないと,大きな問題が発生し,重要なファイルが上書きされる可能性があります.それを避けるために,\textbf{常に中括弧に入れます}.

\begin{decl}
|\setcounter{errorcontextlines}| \arg{num}
\end{decl}

%\TeX~3 introduced a new primitive |\errorcontextlines| which controls
%the format of error messages. \LaTeXe\ provides an interface to this
%through the standard |\setcounter| command. As most \LaTeX\ users do
%not want to see the internal definitions of \LaTeX\ commands each time
%they make an error, \LaTeXe{} sets this to $-1$ by default.
\TeX~3で新しいプリミティブ|\errorcontextlines|を導入しました.
これはエラーメッセージのフォーマットを管理します.
\LaTeXe{}は標準の|\setcounter|コマンドを通して,|\errorcontextlines|とのインターフェースを提供します.
ほとんどの\LaTeX{}ユーザは,エラーが出たからといって\LaTeX{}コマンドの内部の定義を見ようとは思わないでしょう.ですから\LaTeXe{}はデフォルトで$-1$にしています.

%\subsection{Document structure}
\subsection{ドキュメントの構造}

%The |book| document class introduces new commands to indicate
%document structure.
ドキュメントクラスの|book|には,ドキュメントの構造のための新しいコマンド
\begin{decl}
|\frontmatter| \\ |\mainmatter| \\ |\backmatter|
\end{decl}
が導入されました.

%These commands indicate the beginning of the front matter (title page,
%table of contents and prefaces), main matter (main text) and back
%matter (bibliography, indexes and colophon).
これらのコマンドは,それぞれ|\frontmatter|で前付け(扉,目次,序文),|\mainmatter|で本文,|\backmatter|で後付け(参考文献,索引,奥付)であることを明確に示します.

%\subsection{Definitions}
\subsection{定義}

%In \LaTeX, commands can have both mandatory and optional arguments,
%for example in:
\LaTeX{}では,コマンドはは必須の引数とオプションの引数の両方をもつことができます.
たとえば
\begin{verbatim}
   \documentclass[11pt]{article}
\end{verbatim}
%the |11pt| argument is optional, whereas the |article| class name is
%mandatory.
では,|11pt|という引数がオプションですが,|article|クラスという名前は必須です.

%In \LaTeX~2.09 users could define commands with arguments, but these
%had to be mandatory arguments.  With \LaTeXe, users can now define
%commands and environments which also have one optional argument.
\LaTeX~2.09ユーザは引数をもつコマンドを定義していましたが,それらは
必須の引数でなければなりません.
\LaTeXe{}ユーザの場合は,コマンドと環境の両方でオプションの引数を定義できます.

\begin{decl}
|\newcommand| \arg{cmd} \oarg{num} \oarg{default} \arg{definition} \\
|\newcommand*| \arg{cmd} \oarg{num} \oarg{default} \arg{definition} \\
|\renewcommand| \arg{cmd} \oarg{num} \oarg{default} \arg{definition} \\
|\renewcommand*| \arg{cmd} \oarg{num} \oarg{default} \arg{definition}
\end{decl}

%These commands have a new, second, optional argument; this is used for
%defining commands which themselves take one optional argument.  This
%new argument is best introduced by means of a simple (and hence not
%very practical) example:
これらのコマンドには,新しい,つまり2番目のオプション引数があります.
これは1つのオプションの引数を取るコマンドを定義するために使用されます.
この新しい引数について,単純な(したがってそれほど実用的ではない)例題によって説明します.
\begin{verbatim}
   \newcommand{\example}[2][YYY]{Mandatory arg: #2;
                                 Optional arg: #1.}
\end{verbatim}
%This defines |\example| to be a command with two arguments, referred
%to as |#1| and |#2| in the \arg{definition}---nothing new so far.  But
%by adding a second optional argument to this |\newcommand| (the
%|[YYY]|) the first argument (|#1|) of the newly defined command
%|\example| is made optional with its default value being |YYY|.
これは|\example|を,\arg{definition}の中で|#1|と|#2|で示される2つの引数をもつコマンドとして定義します.
これは珍しいことではありません.
しかし,この|\newcommand|(|[YYY]|)に第2のオプションの引数を追加することによって,新しく定義されたコマンド|\example|の最初の引数(|#1|)は省略可能になり,そのデフォルト値は |YYY|になります.

%Thus the usage of |\example| is either:
つまり|\example|の使用結果は
\begin{verbatim}
   \example{BBB}
\end{verbatim}
%which prints:
とすれば,
\begin{quote}
   Mandatory arg: BBB;
   Optional arg: YYY.
\end{quote}
%or:
とプリントされるか,あるは,
\begin{verbatim}
   \example[XXX]{AAA}
\end{verbatim}
%which prints:
とすれば,
\begin{quote}
   Mandatory arg: AAA;
   Optional arg: XXX.
\end{quote}
とプリントされます.

%The default value of the optional argument is \texttt{YYY}.
%This value is specified as the \oarg{default} argument of the
%|\newcommand| that created |\example|.
オプション引数のデフォルトの値は\texttt{YYY}です.
この値は|\example|を作る|\newcommand|の\oarg{default}引数で決まります.

%As another more useful example, the definition:
別の,もっと便利な例は,次の定義
\begin{verbatim}
   \newcommand{\seq}[2][n]{\lbrace #2_{0},\ldots,\,#2_{#1} \rbrace}
\end{verbatim}
%means that the input |$\seq{a}$| produces
%the formula $\lbrace a_{0},\ldots,\,a_{n} \rbrace$,
%whereas the input |$\seq[k]{x}$| produces the formula
%$\lbrace x_{0},\ldots,\,x_{k} \rbrace$.
です.これの入力|$\seq{a}$|は数式$\lbrace a_{0},\ldots,\,a_{n} \rbrace$が生成されます.
一方で,入力|$\seq[k]{x}$|では$\lbrace x_{0},\ldots,\,x_{k} \rbrace$が生成されます.

%In summary, the command:
まとめますと,コマンド
\begin{quote}
   |\newcommand| \arg{cmd} \oarg{num} \oarg{default} \arg{definition}
\end{quote}
%defines \m{cmd} to be a command with \m{num} arguments, the first of
%which is optional and has default value \m{default}.
は,\m{cmd}を引数 \m{num}をもつコマンドとして定義します.
最初の引数はオプションで,デフォルトの値は\m{default}です.

%Note that there can only be one optional argument but, as before,
%there can be up to nine arguments in total.
注意しなければならないのは,ここの例はオプション引数が1つだけであることです.
以前と同じように,引数は9個までもつことができます.

\begin{decl}
|\newenvironment|
 \arg{cmd} \oarg{num} \oarg{default} \arg{beg-def} \arg{end-def} \\
|\newenvironment*|
 \arg{cmd} \oarg{num} \oarg{default} \arg{beg-def} \arg{end-def} \\
|\renewenvironment|
 \arg{cmd} \oarg{num} \oarg{default} \arg{beg-def} \arg{end-def} \\
|\renewenvironment*|
 \arg{cmd} \oarg{num} \oarg{default} \arg{beg-def} \arg{end-def}
\end{decl}

%\LaTeXe\ also supports the creation of environments that have one
%optional argument.  Thus the syntax of these two commands has
%been extended in the same way as that of |\newcommand|.
\LaTeXe{}は,1つのオプション引数を、もつ環境を作ることもサポートしています.
2つのコマンドのシンタックスは,|\newcommand|と同じ方針で拡張されています.

\begin{decl}
|\providecommand| \arg{cmd} \oarg{num} \oarg{default} \arg{definition} \\
|\providecommand*| \arg{cmd} \oarg{num} \oarg{default} \arg{definition}
\end{decl}

%This takes the same arguments as |\newcommand|. If \m{cmd} is already
%defined then the existing definition is kept; but if it is currently
%undefined then the effect of |\providecommand| is to define \m{cmd}
%just as if |\newcommand| had been used.
これは|\newcommand|と同じ引数を取ります.
\m{cmd}が定義済みならば,存在している定義はそのまま使われます.
しかし未定義なら,|\providecommand|の効果は,|\newcommand|で行ったように,\m{cmd}を定義しておかなければなりません.

\NEWfeature{1994/12/01}
%  All the above five `defining commands' now have \texttt{*}-forms that
%  are usually the better form to use when defining commands with
%  arguments, unless any of these arguments is intended to contain
%  whole paragraphs of text.  Moreover, if you ever do find yourself
%  needing to use the non-star form then you should ask whether
%  that argument would not better be treated as the contents of a
%  suitably defined environment.
上記の5つの`定義コマンド'はすべて,テキストの全段落を含むことが意図されていない限り,通常,引数付きのコマンドを定義するときに使用される形式である星印\texttt{*}形式をもちます.
さらに,星印のない形式を使用する必要がある場合は,その引数が適切に定義された環境の内容として扱われるのが良いかどうかを調べる必要があります.

\NEWfeature{1995/12/01}
%  The commands produced by the above five `defining commands' are
%  now robust.
上記で示した5つの`定義コマンド'によって生成されたコマンドは,今では堅牢(robust)になりました.

%\subsection{Boxes}
\subsection{ボックス}

%These next three commands for making LR-boxes all existed in
%\LaTeX~2.09.  They have been enhanced in two ways.
次に示す3つのコマンドはLR-boxes作成のためのもので\LaTeX~2.09に存在していました.
これらは,次の方法で拡張されました.

\begin{decl}
|\makebox| \oarg{width} \oarg{pos}  \arg{text} \\
|\framebox| \oarg{width} \oarg{pos}  \arg{text} \\
|\savebox| \arg{cmd} \oarg{width} \oarg{pos}  \arg{text}
\end{decl}

%One small but far-reaching change for \LaTeXe\ is that, within the
%\m{width} argument only, four special lengths can be used.  These are
%all dimensions of the box that would be produced by using simply
%|\mbox|\arg{text}:
\LaTeXe{}での小さくみえて,とても大きな変化の1つは,\m{width}引数の中だけで,4つの特別な長さを使用できることです.
これらはボックスの次元をもち,単に|\mbox|\arg{text}を使って生成されます.

\begin{itemize}
\item []   |\height|\quad its height above the baseline;
\item []   |\depth|\quad its depth below the baseline;
\item []   |\totalheight|\quad the sum of |\height| and |\depth|;
\item []   |\width|\quad its width.
\end{itemize}
%Thus, to put `hello' in the centre of a box of twice its natural
%width, you would use:
したがって,自然な長さ2つ分のボックスの中心に`hello' を置きたい場合は,
\begin{verbatim}
   \makebox[2\width]{hello}
\end{verbatim}
と書きたいでしょう.
%Or you could put \textit{f} into a square box, like
%this:
あるいは,四角形のボックスの中に\textit{f}を
   \framebox{\makebox[\totalheight]{\itshape f\/}}
\begin{verbatim}
   \framebox{\makebox[\totalheight]{\itshape f\/}}
\end{verbatim}
のようにして入れることもできます.
%Note that it is the total width of the framed box, including the
%frame, which is set to |\totalheight|.
注意しなければならないのは,枠付きのボックスの枠も含めた全部の長さは,|\totalheight|に設定されていることです.

%The other change is a new possibility for \m{pos}: |s| has been added
%to |l| and |r|.  If \m{pos} is |s| then the text is stretched the full
%length of the box, making use of any `rubber lengths' (including any
%inter-word spaces) in the contents of the box.  If no such `rubber
%length' is present, an `underfull box' will probably be produced.
もう1つの変更は,\m{pos}の新しい可能性です.
|l|と|r|に加えて|s|が追加されました.
\m{pos} が|s|ならば, テキストはボックスの長さにあわせて引き伸ばされ,ボックスの中身には任意の`ゴム長さ(rubber lengths)'(ここで単語の間のスペースを含みます)が使用されます.
そのような`ゴム長さ'が与えられていない場合は`underfull box' という表示が出ます.

\begin{decl}
|\parbox| \oarg{pos} \oarg{height} \oarg{inner-pos} \arg{width}
         \arg{text} \\
|\begin{minipage}|
         \oarg{pos} \oarg{height} \oarg{inner-pos} \arg{width}\\
\m{text}\\
|\end{minipage}|
\end{decl}

%As for the box commands above, |\height|, |\width|, etc.~may be used
%in the \oarg{height} argument to denote the natural dimensions of the
%box.
上で示したボックスコマンドと同じような,|\height|,|\width|などが
ボックスの自然な長さを表すために\oarg{height}引数で使われます.

%The \m{inner-pos} argument is new in \LaTeXe.  It is the vertical
%equivalent to the \m{pos} argument for |\makebox|, etc, determining
%the position of \m{text} within the box.  The \m{inner-pos} may be any
%one of |t|, |b|, |c|, or |s|, denoting top, bottom, centered, or
%`stretched' alignment respectively.  When the \m{inner-pos} argument
%is not specified, \LaTeX\ gives it same  value as \m{pos} (this could be
%the latter's default value).
\LaTeXe{}では\m{inner-pos}引数が新しくなりました.
|\makebox|などの\m{pos}引数に垂直で,ボックス内の\m{text} の位置を決定します.
\m{inner-pos}は,|t|,|b|,|c|,あるいは|s|のいずれか1つであり,それぞれ上,下,中央, `引伸'して一揃えをします.
\m{inner-pos}引数が指定されていない場合,\LaTeX{}は\m{pos}と同じ値を与えます(これは後者のデフォルト値です).

\begin{decl}
|\begin{lrbox}| \arg{cmd}\\
\m{text}\\
|\end{lrbox}|
\end{decl}

%This is an environment which does not directly print anything.
%Its effect is to save the typeset \m{text} in the bin \m{cmd}. Thus
%it is like |\sbox| \arg{cmd} \arg{text}, except that any white space
%before or after the contents \m{text} is ignored.
これは何も直接プリントしない環境の一つです.
その効果は,\\m{text} のタイプセットを\m{cmd}に保存することです.
したがって,|\sbox| \arg{cmd} \arg{text}のようなものでが,\m{text}の中身の前後の空白は無視されます.

%This is very useful as it enables both the |\verb| command and the
%\texttt{verbatim} environment to be used within \m{text}.
これは,\m{text}の中で|\verb|コマンドと\texttt{verbatim}環境を使うことができるので便利です.

%It also makes it possible to define, for example, a `framed box'
%environment.  This is done by first using this environment to save
%some text in a bin \m{cmd} and then calling
%|\fbox{\usebox{|\m{cmd}|}}|.
これは,たとえば`framed box'環境を定義することにも使うことができます.
最初にこの環境を使って\m{cmd}にテキストを保存して,そしてから|\fbox{\usebox{|\m{cmd}|}}|を呼び出します.

%The following example defines an environment, called |fmpage|, that is
%a framed version of |minipage|.
次の例では,|fmpage|を呼び出して環境を定義しています.
これは|minipage|の枠付きバージョンです.
\begin{verbatim}
   \newsavebox{\fmbox}
   \newenvironment{fmpage}[1]
     {\begin{lrbox}{\fmbox}\begin{minipage}{#1}}
     {\end{minipage}\end{lrbox}\fbox{\usebox{\fmbox}}}
\end{verbatim}


%\subsection{Measuring things}
\subsection{長さを測る}

%The first of these next commands was in \LaTeX~2.09. The two new
%commands are the obvious analogues.
最初のコマンドは\LaTeX~2.09にあったものです.
新しい2つのコマンドが何かは明らかでしょう.

\begin{decl}
|\settowidth|  \arg{length-cmd} \arg{lr text} \\
|\settoheight| \arg{length-cmd} \arg{lr text} \\
|\settodepth|  \arg{length-cmd} \arg{lr text}
\end{decl}

%\subsection{Line endings}
\subsection{線の終わり}

\NEWdescription{1994/12/01}
%The command |\\|, which is used to indicate a line-end in various
%places, is now a robust command when used within arguments such as
%section titles.
コマンド|\\|は,様々なところで行の終わりを示すの使われますが,節のタイトルに使われる時になどに,堅牢なコマンドになりました.

%Also, because it is often necessary to distinguish which type of line
%is to be ended, we have introduced the following new command; it
%has the same argument syntax as that of |\\|.
どのような種類の線が終わるのかということを知る必要があります.
そこで,次の新しいコマンドを導入しました.
これらも|\\|と同じ引数を取ります.
\begin{decl}[1994/12/01]
|\tabularnewline| \oarg{vertical-space}
\end{decl}
%One example of its use is when the text in the last
%column of a |tabular| environment is set with |\raggedright|; then
%|\tabularnewline| can be used to indicate the end of a row of the
%|tabular|, whilst |\\| will indicate the end of a line of text in a
%paragraph within the column. This command can be used in the |array|
%environment as well as |tabular|, and also the extended versions of
%these environments offered by the \textsf{array} and \textsf{longtable}
%packages in the tools collection.
使用例の一つとして,|tabular|環境の最後の列のテキスト|\raggedright|で設定されています.
そして,|\tabularnewline|は|tabular|の行の終わりを示すために使用できますが,|\\|は列内の段落内のテキスト行の終わりを示します.
このコマンドは|array|環境でも|tabular|と同じように使うことができます.
さらに,この環境の拡張版が\textsf{array}パッケージと\textsf{longtable}パッケージとして,toolsに含まれています.


%\subsection{Controlling page breaks}
\subsection{改ページの制御}

%Sometimes it is necessary, for a final version of a document, to
%`help' \LaTeX\ break the pages in the best way. \LaTeX~2.09 had a
%variety of commands for this situation: |\clearpage|, |\pagebreak|
%etc.  \LaTeXe\ provides, in addition, commands which can produce
%longer pages as well as shorter ones.
ときどき必要となることは、ドキュメントの最終バージョンで\LaTeX{}が望ましい箇所で改ページをするのを`助けてあげる'ことです.
\LaTeX~2.09にはこのような状況のためのコマンドがありました.
それらは|\clearpage|, |\pagebreak|などです.
\LaTeXe{}では,これらに加えて,長いページだけでなく短いページを生成できるコマンドがあります.

\begin{decl}
|\enlargethispage| \arg{size} \\
|\enlargethispage*| \arg{size}
\end{decl}

%These commands increase the height of a page (from its normal value of
%|\textheight|) by the specified amount \m{size}, a rigid length. This
%change affects \emph{only} the current page.
これらのコマンドはページの高さ(標準値は|\textheight|)を\m{size}で指定した量だけ増やします.
この変更は現在のページ\emph{だけ}に反映します.

%This can be used, for example, to allow an extra line to be fitted
%onto the page or, with a negative length, to produce a page
%shorter than normal.
これを使うと,例えば,ページに収まるように余分な線を引いたり,負の長さにして通常のページより狭くできます.

%The star form also shrinks any vertical white space on the page as
%much as possible, so as to fit the maximum amount of text on the
%page.
星印付きも,テキストを1ページに収まるように可能な限り空白行を縮めます.

\NEWdescription{1995/12/01}
%These commands do not change the position of the footer text; thus, if
%a page is lengthened too far, the main text may overprint the footer.
これらのコマンドは,脚注のテキストの位置を変更しません.
したがって,テキストの量(行数)が多いと脚注と重なってしまいます.

%\subsection{Floats}
\subsection{フロート}

%There is a new command, |\suppressfloats|, and a new `float
%specifier'.  These will enable people to gain better
%control of \LaTeX's float placement algorithm.
新しいコマンド|\suppressfloats|と新しい`フロート指示'があります。
これらは\LaTeX{}のフロートを配置するアルゴリズムを、さらによく管理するために使われます。

\begin{decl}
|\suppressfloats| \oarg{placement}
\end{decl}

%This command stops any further floating environments from being placed
%on the current page. With an optional argument, which should be either
%|t| or |b| (not both), this restriction applies only to putting
%further floats at the top or at the bottom.  Any floats which would
%normally be placed on this page are placed on the next page instead.
このコマンドは、それ以上のフローティング環境が現在のページに置かれないようにします。
オプション引数|t|あるいは|b|(どちらか一つ)で制限すれば、フロートをページの上または下に置きます。
通常ならこのページに配置されるフロートは、次のページに配置されます。

\begin{decl}
The extra float location specifier: \ \texttt{!}
\end{decl}

%This can be used, along with at least one of \texttt{h}, \texttt{t},
%\texttt{b} and \texttt{p}, in the location optional argument of a
%float.
フロートの位置決めオプションには、\texttt{h}、\texttt{t}、\texttt{b}と\texttt{p}から少なくとも一つを指定します。

%If a \texttt{!} is present then, just for this particular float,
%whenever it is processed by the float mechanism the following are
%ignored:
\texttt{!}が与えられていたら、この特別なフロートは次のようなことは無視されます。
\begin{itemize}
\item  % all restrictions on the number of floats which can appear;
フロートの数についてのすべての制限、
\item % all explicit restrictions on the amount of space on a text page
%which may be occupied by floats or must be occupied by text.
テキストページのスペースの量に関する明示的な制限は、
フロートかテキストで埋められます。
\end{itemize}
%The mechanism will, however, still attempt to ensure that pages are
%not overfull and that floats of the same type are printed in the
%correct order.
このメカニズムは、このような状況でもページがオーバーフルにならないように努力し、
同じタイプのフロートは正しい順序で印刷されるようにします。

%Note that its presence has no effect on the production of float pages.
これがフロートのページの生成には影響を与えないことに注意してください。

%A \texttt{!} specifier overrides the effect of any |\suppressfloats|
%command for this particular float.
\texttt{!}指定子は,この特別なフロートについての|\suppressfloats|コマンドを無視します。

%\subsection{Font changing: text}
\subsection{フォントの変更:テキスト}

%The font selection scheme used in \LaTeXe{} differs a lot from that
%used in \LaTeX~2.09.  In this section, we give a brief description of
%the new commands.  A more detailed description with examples is given
%in \emph{\LaTeXcomp}, and the interface for class- and package-writers
%is described in \emph{\fntguide}.
\LaTeXe{}で使用されているフォント選択スキームは、\LaTeX~2.09で使用されているフォント選択スキームと異なることが多いです。この節では、新しいコマンドについて簡単に説明します。
詳しい説明と実例については\emph{\LaTeXcomp}にあります。クラスととパッケージの製作者向けのインターフェースは\emph{\fntguide}で説明されています。

\begin{decl}
|\rmfamily|\\
|\sffamily|\\
|\ttfamily|\\
|\mdseries|\\
|\bfseries|\\
|\upshape|\\
|\itshape|\\
|\slshape|\\
|\scshape|
\end{decl}

%These are font commands whose use is the same as the commands |\rm|,
%|\bf|, etc.  The difference is that each command changes just one
%attribute of the font (the attribute changed is part of the name).
%One result of this is that, for example, |\bfseries\itshape| produces
%both a change of series and a change of shape, to give a bold italic
%font.
これらはフォントコマンドで、|rm|、|bf|などと同じです。
違いは、これらのコマンドではフォントの1つの属性だけを変更することです(変更された属性は名前の一部です)。
これによる結果の1つは、たとえば、|\bfseries\itshape|はシリーズの変更と字形の変化の両方を行うので、太字のイタリック体を与えます。

\begin{decl}
|\textrm|\arg{text}\\
|\textsf|\arg{text}\\
|\texttt|\arg{text}\\
|\textmd|\arg{text}\\
|\textbf|\arg{text}\\
|\textup|\arg{text}\\
|\textit|\arg{text}\\
|\textsl|\arg{text}\\
|\textsc|\arg{text}\\
|\emph|\arg{text}
\end{decl}

%These are one-argument commands; they take as an argument the
%text which is to be typeset in the particular font. They also
%automatically insert italic corrections where appropriate; if you do
%not like the result, you can add an italic correction with |\/| or
%remove it with |\nocorr|.  The |\nocorr| should always be the first or
%last thing within the \arg{text} argument.
これらは引数を1つだけもつコマンドです。
特定のフォントでタイプセットさせるテキストを引数として取ります。
必要であれば斜体補正を自動的に挿入します。その結果が気に入らなければ、|\/|でイタリック補正を加えることも、|nocorr|で削除することもできます。
|\nocorr|は\arg{text}引数の中で最初か最後に置かなければなりません。

%\subsection{Font changing: math}
\subsection{フォントエンコーディング:数式}

%Most of the fonts used within math mode do not need to be explicitly
%invoked; but to use letters from a range of fonts, the following
%class of commands is provided.
数学モードで使用されるフォントのほとんどは、明示的に呼び出す必要はありません。
使うことができるフォントから文字を使用するために、以下のコマンド類が用意されています。

\begin{decl}
|\mathrm| \arg{letters}\\
|\mathnormal| \arg{letters}\\
|\mathcal| \arg{letters}\\
|\mathbf| \arg{letters}\\
|\mathsf| \arg{letters}\\
|\mathtt| \arg{letters}\\
|\mathit| \arg{letters}
\end{decl}

%These are also one-argument commands which take as an argument the
%letters which are to be typeset in the particular font.  The argument
%is processed in math mode so spaces within it will be ignored.  Only
%letters, digits and accents have their font changed, for example
%|$\mathbf{\tilde A \times 1}$| produces $\mathbf{\tilde A \times 1}$.
さらに引数として文字を1つだけとるコマンドがあり、それは特定のフォントでタイプセットします。
引数は数学モードで処理されるので、その中のスペースは無視されます。
文字、数字、そしてアクセントだけが変更されます。たとえば、|$\mathbf{\tilde A \times 1}$|とすれば$\mathbf{\tilde A \times 1}$が生成されます。

%\subsection{Ensuring math mode}
\subsection{強制数式モード}

\begin{decl}
|\ensuremath| \arg{math commands}
\end{decl}

%In \LaTeX~2.09, if you wanted a command to work both in math mode and
%in text mode, the suggested method was to define something like:
\LaTeX~2.09では、数寿司気モードでもテキストモードでも働くコマンドが必要な場合は、
次のようにすることでした。

\begin{verbatim}
   \newcommand{\Gp}{\mbox{$G_p$}}
\end{verbatim}
%Unfortunately, the |\mbox| stops |\Gp| changing size correctly in (for
%instance) subscripts or a fraction.
残念ですが、|\mbox|は|\Gp|が添え字や分数に現れる時に、正しい大きさになりません。

%In \LaTeXe{} you can define it thus:
\LaTeXe{}では、次のようにします。
\begin{verbatim}
   \newcommand{\Gp}{\ensuremath{G_p}}
\end{verbatim}
%Now |\Gp| will work correctly in all contexts.
こうすれば、|\Gp|は、どの文脈でも正しく働きます。

%This is because the |\ensuremath| does nothing, producing simply
%|G_p|, when |\Gp| is used within math mode; but it ensures that math
%mode is entered (and exited) as required when |\Gp| is used in text
%mode.
これが意味することは、|\ensuremath|は何もせずに、|\Gp|が数式モードで呼ばれたら単に|G_p|を生成します。しかし|\Gp|がテキストモードで呼ばれたとき、必要ならば数式モードに入り(処理が終わったら出)ます。

%\subsection{Setting text superscripts}
\subsection{本文での添字の設定}

\begin{decl}
|\textsuperscript| \arg{text}
\end{decl}

\NEWfeature{1995/06/01} % In \LaTeX~2.09 textual superscripts such as
%footnote markers were produced by internally entering math mode and
%typesetting the number as a math superscript.  This normally looked
%fine since the digits in math fonts are the same as those in text
%fonts when Computer Modern fonts are used.  But when a different
%document font (such as Times) is selected, the results look rather
%strange.  For this reason the command |\textsuperscript| has been
%introduced which typesets its argument in the current text font, in a
%superscript position and in the correct size.
\LaTeX~2.09では、脚注記号などのテキストで使われる上付き文字は、内部的に数学モードに入り、数字を数式の上付き文字として生成されました。
コンピュータモダンフォントが使用されている場合、数学フォントとテキストフォントの数字は同じであるため、通常は正しい見た目となります。
しかし、異なるドキュメントフォント(たとえばTimesなど)を選択すると、結果の見ためはおかしくなります。
このため、|\textsuperscript|コマンドが用意されました。これは引数を現在のテキストフォントを使って上付き文字の位置に正しいサイズでタイプセットします。

%\subsection{Text commands: all encodings}
\subsection{テキストコマンド:すべてのエンコーディング}

\NEWdescription{1994/12/01}
%  One of the main differences between \LaTeXe{} and \LaTeX~2.09 is
%  that \LaTeXe{} can deal with fonts in arbitrary \emph{encodings}.
%  (A font encoding is the sequence of characters in the font---for
%  example a Cyrillic font would have a different encoding from a Greek
%  font.)
\LaTeXe{}と\LaTeX~2.09の主な違いの1つは、\LaTeXe{}が任意の\emph{エンコーディング}のフォントを扱うことができることです。(フォントエンコーディングはフォント内の文字の並びです。たとえば、キリル文字フォントはギリシア語フォントとは異なるエンコーディングになっています。)

%  The two major font encodings that are used for Latin languages such
%  as English or German are |OT1| (Donald Knuth's 7-bit encoding, which
%  has been used during most of \TeX's lifetime) and |T1| (the new
%  8-bit `Cork' encoding).
英語やドイツ語などのラテン語に使用される2つの主要なフォントエンコーディングは、|OT1|(Donald Knuthの7ビットエンコーディングです。これは、長い間\TeX{}で使用されていました)と|T1|(8ビットの `Cork 'エンコーディングです)。

%  \LaTeX~2.09 only supported the |OT1| encoding, whereas \LaTeXe{} has
%  support for both |OT1| and |T1| built-in.  The next section will
%  cover the new commands which are available if you have |T1|-encoded
%  fonts.  This section describes new commands which are available in
%  all encodings.
\LaTeX~2.09は|OT1|エンコードのみがサポートされていますが、\LaTeXe{}は|OT1|と|T1|のどちらもサポートされています。
次の節では、|T1|でエンコードされたフォントを使用する時に使う新しいコマンドについて説明します。
この節では、すべてのエンコーディングで使用できる新しいコマンドについて説明します。

%  Most of these commands provide characters which were available in
%  \LaTeX~2.09 already.  For example |\textemdash| gives an `em dash',
%  which was available in \LaTeX~2.09 by typing |---|.  However, some
%  fonts (for example a Greek font) may not have the |---| ligature,
%  but you will still be able to access an em dash by typing
%  |\textemdash|.
これらのコマンドのほとんどは\LaTeX~2.09でも利用可能な文字を提供しています。
たとえば、|\textemdash|は `エムダッシュ 'を与えますが、これは \LaTeX~2.09では| --- |のようにタイプしていました。
しかし、いくつかのフォント(たとえばギリシア語のフォントなど)では、|---|のためのリガチャがありませんが|\textemdash |と入力すればエムダッシュを生成できます。

\begin{decl}[1994/12/01]
      |\r{<text>}|
\end{decl}
%   This command gives a `ring' accent, for example `\r{o}' can be typed
%   |\r{o}|.
このコマンドは`輪っか'のアクセントを与えます。たとえば、`\r{o}'を生成にするには|\r{o}|というようにタイプします。

\begin{decl}[1994/12/01]
      |\SS|
\end{decl}
%  This command produces a German `SS', that is a capital `\ss'.  This
%  letter can hyphenate differently from `SS', so is needed for entering
%  all-caps German.
このコマンドはドイツ語の`SS'、つまり大文字の`\ss'を生成します。
この文字の綴りは`SS'ではないので、大文字で入力します。

\begin{decl}[1994/12/01]
      |\textcircled{<text>}|
\end{decl}
%   This command is used to build `circled characters' such as
%   |\copyright|.  For example |\textcircled{a}| produces
%   \textcircled{a}.
このコマンドは|\copyright|のような`丸囲み文字'を作るのに使われます。
たとえば|\textcircled{a}|とすれば\textcircled{a}が生成されます。

\begin{decl}[1994/12/01]
      |\textcompwordmark|
\end{decl}
%  This command is used to separate letters which would normally
%  ligature.  For example `f\textcompwordmark i' is produced with
%  |f\textcompwordmark i|.  Note that the `f' and `i' have not
%  ligatured to produce `fi'.  This is rarely useful in English
%  (`shelf\textcompwordmark ful' is a rare example of where it might be
%  used) but is used in languages such as German.
このコマンドは通常なら理ガチャされる文字を話して生成するために使われます。
たとえば、|f\textcompwordmark i|とすれば、`f\textcompwordmark i'となります。
`f'と`i'は、 `fi'となりません。
これは英語では、滅多に必要とはなりません(`shelf\textcompwordmark ful'が、稀な例です)が、ドイツ語などでは必要となります。

\begin{decl}[1994/12/01]
      |\textvisiblespace|
\end{decl}
%   This command produces a `visible space' character
%   `\textvisiblespace'.  This is sometimes used in computer listings,
%   for example `type \textsf{hello\textvisiblespace world}'.
このコマンドは`見える空白'文字`\textvisiblespace'を生成します。
これはコンピュータのプログラムリストなどで、たとえば
`type \textsf{hello\textvisiblespace world}'のように使われます。

\begin{decl}[1994/12/01]
      |\textemdash|
      |\textendash|
      |\textexclamdown|
      |\textquestiondown| \\
      |\textquotedblleft|
      |\textquotedblright|
      |\textquoteleft|
      |\textquoteright|
\end{decl}
%   These commands produce characters which would otherwise be
%   accessed via ligatures:
これらのコマンドは特殊な文字の生成かリガチャ(合字)させたいときに使われます。
%   \begin{center}
%      \begin{tabular}{ccl}
%         \emph{ligature} & \emph{character} & \emph{command} \\
%         |---| & --- & |\textemdash| \\
%         |--|  & --  & |\textendash| \\
%         |!`|  & !`  & |\textexclamdown| \\
%         |?`|  & ?`  & |\textquestiondown| \\
%         |``|  & ``  & |\textquotedblleft| \\
%         |''|  & ''  & |\textquotedblright| \\
%         |`|   & `   & |\textquoteleft| \\
%         |'|   & '   & |\textquoteright|
%      \end{tabular}
%   \end{center}
      \begin{center}
      \begin{tabular}{ccl}
         \emph{リガチャ} & \emph{文 字} & \emph{コマンド} \\
         |---| & --- & |\textemdash| \\
         |--|  & --  & |\textendash| \\
         |!`|  & !`  & |\textexclamdown| \\
         |?`|  & ?`  & |\textquestiondown| \\
         |``|  & ``  & |\textquotedblleft| \\
         |''|  & ''  & |\textquotedblright| \\
         |`|   & `   & |\textquoteleft| \\
         |'|   & '   & |\textquoteright|
      \end{tabular}
   \end{center}
%   The reason for making these characters directly accessible is so
%   that they will work in encodings which do not have these characters.
これらの文字を作る理由は、そのような文字をもっていないエンコーディングでも生成できるようにするためです。

\begin{decl}[1994/12/01]
      |\textbullet|
      |\textperiodcentered|
\end{decl}
%   These commands allow access to characters which were previously only
%   available in math mode:
これらのコマンドは、これまで数式モードでのみ使われる文字をテキストで使うためです。
%   \begin{center}
%      \begin{tabular}{lcl}
%         \emph{math command} & \emph{character} & \emph{text command} \\
%          |\bullet|   & $\bullet$   & |\textbullet| \\
%         |\cdot|     & $\cdot$     & |\textperiodcentered|
%      \end{tabular}
%   \end{center}
      \begin{center}
      \begin{tabular}{lcl}
         \emph{数式コマンド} & \emph{文 字} & \emph{テキストコマンド} \\
          |\bullet|   & $\bullet$   & |\textbullet| \\
         |\cdot|     & $\cdot$     & |\textperiodcentered|
      \end{tabular}
   \end{center}

\begin{decl}[1995/12/01]
      |\textbackslash|
      |\textbar|
      |\textless|
      |\textgreater|
\end{decl}
%   These commands allow access to ASCII characters which were
%   only available in verbatim or math mode:
これらのコマンドは、verbatimモードまたは数式モードでのみ使われるASCII文字を使うためです。
%   \begin{center}
%      \begin{tabular}{lcl}
%         \emph{math command} & \emph{character} & \emph{text command} \\
%          |\backslash|   & $\backslash$   & |\textbackslash| \\
%          |\mid|     & $\mid$     & |\textbar| \\
%          |<<| & $<$ & |\textless| \\
%          |>>| & $>$ & |\textgreater|
%      \end{tabular}
%   \end{center}
   \begin{center}
      \begin{tabular}{lcl}
         \emph{数式コマンド} & \emph{文 字} & \emph{テキストコマンド} \\
          |\backslash|   & $\backslash$   & |\textbackslash| \\
          |\mid|     & $\mid$     & |\textbar| \\
          |<<| & $<$ & |\textless| \\
          |>>| & $>$ & |\textgreater|
      \end{tabular}
   \end{center}
 
\begin{decl}[1995/12/01]
      |\textasciicircum|
      |\textasciitilde|
\end{decl}
%   These commands allow access to ASCII characters which were
%   previously only available in verbatim:
これらのコマンドは、verbatimモードでのみ使われるASCII文字を使うためです。
%   \begin{center}
%      \begin{tabular}{cl}
%         \emph{verbatim} & \emph{text command} \\
%         |^| & |\textasciicircum| \\
%         |~| & |\textasciitilde|
%      \end{tabular}
%   \end{center}
   \begin{center}
      \begin{tabular}{cl}
         \emph{verbatim} & \emph{テキストコマンド} \\
         |^| & |\textasciicircum| \\
         |~| & |\textasciitilde|
      \end{tabular}
   \end{center}
   
\begin{decl}[1995/12/01]
      |\textregistered|
      |\texttrademark|
\end{decl}
%   These commands provide the `registered trademark' (R) and
%   `trademark' (TM) symbols.
これらのコマンドは`登録商標’(R)記号と`商標'(TM)記号を生成します。

%\subsection{Text commands: the T1 encoding}
\subsection{テキストコマンド:T1エンコーディング}

\NEWdescription{1994/12/01}
%  The |OT1| font encoding is fine for typesetting in English, but has
%  problems when typesetting other languages.  The |T1| encoding solves
%  some of these problems, by providing extra characters (such as `eth'
%  and `thorn'), and it allows words containing accented letters to be
%  hyphenated (as long as you have a package like |babel| which allows
%  for non-American hyphenation).
|OT1|フォントのエンコーディングは英語のタイプセットには問題ありませんが、他の言語のタイプセットには問題があります。
|T1|エンコーディングは、余分な文字(たとえば`eth'と`thorn')を提供することによってこれらの問題のいくつかを解決し、アクセント付きの文字を含む単語のハイフネーション(|babel|のようなパッケージを使えばアメリカ英語以外のハイフネーション)も行えます。

%  This section describes the commands you can use if you have the |T1|
%  fonts.  To use them, you need to get the `ec fonts', or the
%  |T1|-encoded PostScript fonts, as used by \textsf{psnfss}.
%  All these fonts are
%  available by anonymous ftp in the Comprehensive \TeX{} Archive, and
%  are also available on the CD-ROMs \emph{4all \TeX} and
%  \emph{\TeX{} Live} (both available from the \TeX{} Users Group).
この節では、|T1|フォントをもっている場合に、それを使う方法を説明します。
そのためには、`ec fonts'、あるいは`textf {psnfss}で使われているような|T1|でエンコードされたPostScriptフォントが必要です。
これらすべてのフォントはComprehensive\TeX{}アーカイブ(CTAN)の匿名ftpで取り寄せるか、\emph{4all\TeX}、または\emph{\TeX {} Live}というCD-ROMから入手できます。(\TeX{}ユーザーグループから入手できます)。

%   You can then select the |T1| fonts by saying:
これらの準備ができたら、|T1|フォントは、次のようにすれば選ぶことができます。
\begin{verbatim}
   \usepackage[T1]{fontenc}
\end{verbatim}
%   This will allow you to use the commands in this section.
こうすれば、この節で説明しているコマンドが利用できます。

%   \emph{Note:} Since this document must be processable on any site
%   running an up-to-date \LaTeX, it does not contain any characters
%   that are present only in |T1|-encoded fonts.  This means that this
%   document cannot show you what these glyphs look like!  If you want
%   to see them then run \LaTeX{} on the document |fontsmpl| and
%   respond `|cmr|' when it prompts you for a family name.
\emph {注意:}このドキュメントは、最新の\LaTeX{}を実行しているサイトで処理可能にするために、|T1|エンコードされたフォントのみ存在する文字は含みません。
つまり、このドキュメントでは、これらのグリフがどうであるかを表示できません。
それらを見たいなら、|fontsmpl|ドキュメントを\LaTeX{}で実行してください。プロンプト`|cmr| 'が現れたらフォントのファミリーネームを与えてください。

\begin{decl}[1994/12/01]
      |\k{<text>}|
\end{decl}
%   This command produces an `ogonek' accent.
このコマンドはポーランド語などの`オゴネク'アクセントを生成します。

\begin{decl}[1994/12/01]
      |\DH|
      |\DJ|
      |\NG|
      |\TH|
      |\dh|
      |\dj|
      |\ng|
      |\th|
\end{decl}
%   These commands produce characters `eth', `dbar', `eng', and `thorn'.
これらのコマンドは、文字`eth'、`dbar'、`eng'`thorn'を生成します。

\begin{decl}[1994/12/01]
      |\guillemotleft|
      |\guillemotright|
      |\guilsinglleft|
      |\guilsinglright| \\
      |\quotedblbase|
      |\quotesinglbase|
      |\textquotedbl|
\end{decl}
   % A local hack (could be improved):
   \newcommand{\fauxguillemet}[1]{$\vcenter{\hbox{$\scriptscriptstyle#1$}}$}
%   These commands produce various sorts of quotation mark.
%   Rough representations of them are:
これらのコマンドは、いろいろな引用符を生成します。
大まかに表示すれば、次のようななものです。
%   \fauxguillemet\ll a\fauxguillemet\gg{} 
%   \fauxguillemet<a\fauxguillemet> 
%   ,\kern -0.1em,\kern 0.05em a\kern -0.05em`` 
%   ,\kern 0.05em a\kern -0.05em` and |"|a|"|.
   \fauxguillemet\ll a\fauxguillemet\gg{} 
   \fauxguillemet<a\fauxguillemet> 
   ,\kern -0.1em,\kern 0.05em a\kern -0.05em`` 
   ,\kern 0.05em a\kern -0.05em`そして|"|a|"|。

\NEWdescription{2001/06/01}
%   There are therefore some extra short-form ligatures available for
%   use in documents that will only be used with |T1|-encoded fonts.
したがって|T1|エンコードされフォントだけを使っている場合に必要なリガチャが必要となります。
 
%   The guillemets |\guillemotleft| and |\guillemotright|% 
%   \footnote{We apologise once again for maintaining Adobe's 
%     enormous solipsism~(sic) of confusing a diving bird with 
%     punctuation marks!}  
%   can be obtained by typing |<<<<| and |>>>>| and |\quotedblbase| 
%   by typing |,,|\,.  
ギュメ記号|\guillemotleft|と|\guillemotright|%
\footnote{私たちは、アドビの多くの特徴をサポートするために、混乱させたことについて、再度謝ります!}
は、|<<<<|と|>>>>|、そして|\quotedblbase|は|,,|とすれば得られます。

%   Also, unlike the unexpected results with
%   |OT1|-encoded fonts, |<<| and |>>| will produce \textless{} and
%   \textgreater{}.
さらに、予想外の結果ですが、|OT1|エンコードフォント、|<<|と|>>|は\textless{}と
\textgreater{}とすれば得られます。

%   Note also that the single character |"| will no longer produce ''
%   but rather |\textquotedbl|.
さらに注意することは、単一の文字|"|は''を生成しませんが、|\textquotedbl|とすれば得ることができます。

%\subsection{Logos}
\subsection{ロゴ}

\begin{decl}
|\LaTeX|\\
|\LaTeXe|
\end{decl}

%|\LaTeX| (producing `\LaTeX') is still the `main' logo command,
%but if you need to refer to the new features, you can write
%|\LaTeXe| (producing `\LaTeXe').
|\LaTeX|(`\LaTeX'が生成されます)は、現在でも主なロゴコマンドですが、
新しい機能を示したいときには、|\LaTeXe|(`\LaTeXe'が生成されます)とします。

%\subsection{Picture commands}
\subsection{描画コマンド}

\begin{decl}
   |\qbezier[<N>](<AX>,<AY>)(<BX>,<BY>)(<CX>,<CY>)| \\
   | \bezier{<N>}(<AX>,<AY>)(<BX>,<BY>)(<CX>,<CY>)|
\end{decl}
%The |\qbezier| command can be used in |picture| mode to draw a
%quadratic Bezier curve from position |(<AX>,<AY>)| to |(<CX>,<CY>)| with
%control point |(<BX>,<BY>)|.  The optional argument \m{N} gives the
%number of points on the curve.
|\qbezier|コマンドは、|picture|モードで、
|(<AX>,<AY>)|から|(<CX>,<CY>)|までの2次のBezier曲線をコントロール点を|(<BX>,<BY>)|で描くときに使われます。

%For example, the diagram:
たとえば、ダイアグラム
\begin{center}
   \begin{picture}(50,50)
      \thicklines
      \qbezier(0,0)(0,50)(50,50)
      \qbezier[20](0,0)(50,0)(50,50)
      \thinlines
      \put(0,0){\line(1,1){50}}
   \end{picture}
\end{center}
%is drawn with:
は、次のようにして描かれます。
\begin{verbatim}
   \begin{picture}(50,50)
      \thicklines
      \qbezier(0,0)(0,50)(50,50)
      \qbezier[20](0,0)(50,0)(50,50)
      \thinlines
      \put(0,0){\line(1,1){50}}
   \end{picture}
\end{verbatim}
%The |\bezier| command is the same, except that the argument \m{N} is not
%optional.  It is provided for compatibility with the \LaTeX~2.09
%|bezier| document style option.
|\bezier|コマンドは、これまでと同じですが、引数\m{N}はオプションではありません。
これは\LaTeX~2.09の|bezier|ドキュメントのスタイルオプションとの互換性のために与えられています。

%\subsection{Old commands}
\subsection{古いコマンド}

\begin{decl}
|\samepage|
\end{decl}

%The |\samepage| command still exists but is no longer being
%maintained.
%This is because it only ever worked erratically; it does not
%guarantee that there will be no page-breaks within its scope; and
%it can cause footnotes and marginals to be wrongly placed.
|\samepage|コマンドはまだ存在しますが、もはや維持されていません。
このコマンドはうまく機能しなかったためです。対象とする範囲内でページ区切りが発生しないことを保証していません。脚注とマージン(余白)が間違った場所に置かれることがあります。

%We recommend using |\enlargethispage| in conjunction with page-break
%commands such as |\newpage| and |\pagebreak| to help control page
%breaks.
私たちは、これの代わりに|\enlargethispage|をページ区切りコマンド|\newpage|と|\pagebreak|と組み合わせて、ページ区切りを制御することをお勧めします。

\begin{decl}
   |\SLiTeX|
\end{decl}
%Since \SLiTeX{} no longer exists, the logo is no longer defined in the
%\LaTeX{} kernel.  A suitable replacement is |\textsc{Sli\TeX}|.  The
%\SLiTeX{} logo is defined in \LaTeX~2.09 compatibility mode.
\SLiTeX{}は存在しませんので、\LaTeX{}ではロゴは定義されていません。
代わりに|\textsc{Sli\TeX}|を使ってください。
\SLiTeX{}ロゴは、\LaTeX~2.09互換モードでは定義されています。


\begin{decl}
|\mho| |\Join| |\Box| |\Diamond| |\leadsto| \\
|\sqsubset| |\sqsupset| |\lhd| |\unlhd| |\rhd| |\unrhd|
\end{decl}

%These symbols are contained in the \LaTeX{} symbol font, which was
%automatically loaded by \LaTeX~2.09.  However, \TeX{} has room for
%only sixteen math font families; thus many users discovered that they
%ran out.  Because of this, \LaTeX{} does not load the \LaTeX{} symbol
%font unless you use the \textsf{latexsym} package.
これらの記号は、\LaTeX~2.09で自動的に読み込まれる\LaTeX{}シンボルフォントには含まれています。
しかし、\TeX{}は数学フォントファミリを16種しか扱えません。そして多くのユーザーは、それらがなくなったことを発見しました。そのため\LaTeXe{}は\textsf {latexsym}パッケージを使用しない限り、\LaTeX{}記号フォントを読み込みません。

%These symbols are also made available, using different fonts, by the
%\textsf{amsfonts} package, which also defines a large number of other
%symbols.  It is supplied by the American Mathematical Society.
これらの記号は、異なるフォントを使用すれば利用できます。たとえば\textsf{amsfonts}パッケージには、さらに多くの記号が定義されています。これは、アメリカ数学会から提供されています。

%The \textsf{latexsym} package is loaded automatically in \LaTeX~2.09
%compatibility mode.
\textsf{latexsym}パッケージは、\LaTeX~2.09互換モードでは自動的に読み込まれます。

%\section{\LaTeX~2.09 documents}
\section{\LaTeX~2.09ドキュメント}
\label{Sec:209}

%\LaTeXe{} can process (almost) any \LaTeX~2.09 document, by entering
%\emph{\LaTeX~2.09 compatibility mode}.  Nothing has changed, you run
%\LaTeX{} in the same way you always did, and you will get much the
%same results.
\LaTeXe{}はほとんどの\LaTeX~2.09{}ドキュメントを\emph{\LaTeX~2.09互換モード}で処理できます。
何も変更されていないので、\LaTeX{}を同じようなに実行でき、ほとんど同じ結果を得ます。

%The reason for the `almost' is that some \LaTeX~2.09 packages made use
%of low-level unsupported features of \LaTeX.  If you discover such a
%package, you should find out if it has been updated to work with
%\LaTeXe.  Most packages will still work with \LaTeXe---the easiest way
%to find out whether a package still works is to try it!
`ほとんど'という理由は、いくつかの\LaTeX~2.09パッケージは\LaTeX{}がサポートしない低レベルの機能を使っているからです。
そのようなパッケージを見つけたら、\LaTeXe{}でも機能するようになっているか調べてください。
ほとんどのパッケージは、\LaTeXe{}でもそのまま働きます---
どのパッケージが問題ないかを確かめるためには、とにかく試してみることです。

%\LaTeX~2.09 compatibility mode is a comprehensive emulation of
%\LaTeX~2.09, but at the cost of time.  Documents can run up to 50\%
%slower in compatibility mode than they did under \LaTeX~2.09.
%In addition, many of the new features of \LaTeXe{} are not available in
%\LaTeX~2.09 compatibility mode.
\LaTeX~2.09互換モードは、\LaTeX~2.09をエミュレートしますが、その引き換えに時間がかかります。
互換せモードは\LaTeX~2.09に比べて50パーセントほど遅くなります。
その上、\LaTeXe{}の新しい機能は\LaTeX~2.09互換モードにはありません。

%\subsection{Warning}
\subsection{警 告}

\NEWdescription{1995/12/01}
%This \emph{\LaTeX~2.09 compatibility mode} is provided solely to allow
%you to process 2.09 documents, i.e.~documents that were written (we
%hope, a long time ago) for a very old system and therefore could be
%processed by using a genuine antique \LaTeX~2.09 system.
\emph{\LaTeX~2.09互換モード}は、2.09のドキュメントの処理(かなり昔に書かれたものであってほしいのですが)、つまり古いシステムで書かれたためにその時代の骨董的な\LaTeX~2.09を使いたい、のためだけにあります。

%This mode is therefore \emph{not} intended to provide access to the
%enhanced features of \LaTeXe{}.  Thus it must not be used to process
%new documents which masquerade as 2.09 documents (i.e.~they begin with
%|\documentstyle|) but which could not be processed using that genuine
%antique \LaTeX~2.09 system because they contain some new,
%\LaTeXe{}-only, commands or environments.
そのため、このモードは\LaTeXe{}の拡張機能を使うためのものでは\emph{ありません}。
新しいドキュメントを2.09ドキュメントのふりをさせて(つまり|\documentstyle|で始まる)処理すために使うべきではありません。
\LaTeXe{}にだけ備わっているコマンドや環境があるので、骨董的な\LaTeX~2.09システムでは処理できません。

%To prevent such misuse of the system, and the consequent trouble it
%causes when such misleadingly encoded documents are distributed, the
%\emph{\LaTeX~2.09 compatibility mode} turns off most of these new
%features and commands.  Any attempt to use them will give you an error
%message and, moreover, many of them simply  will not work, whilst
%others will produce unpredictable results.
%So don't bother sending us any bug reports about such occurrences since
%they are intentional.
このようなシステムの誤用を防ぐため、そしてまた誤ってエンコードされたドキュメントで引き起こされる問題を回避するために、\emph{LaTeX~2.09互換モード}は新機能やコマンドのほとんどをオフにします。
それらを使用しようとするとエラーメッセージが表示され、さらに多くのものは機能せず、予期しない結果をもたらすことになります。
これは意図的にしたことなので、そのようは現象が発生してもバグ報告を私たちに送らないように。

%\subsection{Font selection problems}
\subsection{フォント選択時の問題}
\label{Sec:fsprob}

%When using compatibility mode, it is possible that you will find
%problems with font-changing commands in some old documents.  These
%problems are of two types:
互換モードを使用すると、古いドキュメントで行っていたフォント変更コマンドに問題が起こることがあります。
これらの問題には2種類があります。

\begin{itemize}
\item producing error messages;
\item not producing the font changes you expected.
\end{itemize}

%In case of error messages it is possible that the document (or an old
%style file used therein) contains references to old internal commands
%which are no longer defined, see Section~\ref{Sec:oldinternals} for
%more information if this is the case.
このエラーメッセージでは、ドキュメント(あるいは古いスタイルファイルが使われている)は、サポートされなくなった古い内部コマンドへの参照がある場合です。
このような場合の情報は\ref{Sec:oldinternals}節にあります。

\NEWdescription{1995/12/01}
%One example of the unexpected is if you use one of the new style of
%math-mode font changing command as follows:
予想しない結果を得る例の一つは、新しい数式モードのフォント変更コマンドを使った時です。
\begin{verbatim}
$ \mathbf{xy} A $
\end{verbatim}
%You may well find that this behaves as if you had put:
次のようにすると
\begin{verbatim}
$ \bf {xy} A $
\end{verbatim}
%everything including the $A$ coming out bold.
$A$も太字になります。

%\LaTeX~2.09 allowed sites to customize their \LaTeX{} installation,
%which resulted in documents producing different results on different
%\LaTeX{} installations.  \LaTeXe{} no longer allows so much
%customization but, for compatibility with old documents, the local
%configuration file |latex209.cfg| is loaded every time \LaTeXe{}
%enters \LaTeX~2.09 compatibility mode.
\LaTeX~2.09は、\LaTeX{}のインストールをカスタマイズすることができたので、
インストールされた\LaTeX{}が異なれば異なる結果が得られました。
\LaTeXe{}は大きなカスタマイズをできないようしましたが、その代わりに古いドキュメントとの互換性のために、ローカルの設定ファイル|latex209.cfg|が\LaTeXe{}が\LaTeX~2.09互換モードに入るたびに読み込まれます。

%For example, if your site was customized to use the New Font
%Selection Scheme (\NFSS) with the |oldlfont| option, then you can
%make \LaTeXe{} emulate this by creating a |latex209.cfg| file
%containing the commands:
たとえば、使用しているシステムが、新しいフォント選択スキーム(\NFSS)を|oldlfont|オプションで使用するようになっていれば、\LaTeXe{}のエミュレートはこれに従って|latex209.cfg|ファイルに
\begin{verbatim}
\ExecuteOptions{oldlfont}\RequirePackage{oldlfont}
\end{verbatim}
が生成されています
%Similarly, to emulate \NFSS{} with the |newlfont| option, you can
%create a |latex209.cfg| file containing:
同じように、|newlfont|オプションで \NFSS{}をエミュレートすると、|latex209.cfg|には
\begin{verbatim}
\ExecuteOptions{newlfont}\RequirePackage{newlfont}
\end{verbatim}
が生成されています。


%\subsection{Native mode}
\subsection{ネイティブモード}
\label{Sec:native}

To run an old document faster, and use the new features of
\LaTeXe, you should try using \emph{\LaTeXe{} native mode}.
This is done by replacing the command:
古いドキュメントの処理を速く処理し、そして\LaTeXe{}
の新しい機能を使いたいのなら\emph{\LaTeXe{}ネテイィブモード}
を使いなさい。
コマンドを置き換えるだけですみます。古い
\begin{quote}
   |\documentstyle[|\m{options}|,|\m{packages}|]|\arg{class}
\end{quote}
%with:
を
\begin{quote}
   |\documentclass|\oarg{options}\arg{class} \\
   |\usepackage{latexsym,|\m{packages}|}|
\end{quote}
とします。
%However, some documents which can be processed in \LaTeX~2.09
%compatibility mode may not work in native mode.  Some \LaTeX~2.09
%packages will only work with \LaTeXe{} in 2.09 compatibility mode.
%Some documents will cause errors because of \LaTeXe's improved error
%detection abilities.
しかし、いくつかのドキュメントはネテイィブモードでは働かず\LaTeX~2.09互換モードでしか処理できないでしょう。
いくつかの\LaTeX~2.09パッケージはLaTeXe{}では\LaTeX~2.09互換モードでのみ動作するでしょう。
いくつかのドキュメント\LaTeXe{}で改良されたエラー発見機能のために、エラーが出るでしょう。

%But most \LaTeX~2.09 documents can be processed by \LaTeXe{}'s native
%mode with the above change.  Again, the easiest way to find out
%whether your documents can be processed in native mode is to try it!
しかし、ほとんどの\LaTeX~2.09ドキュメントは\LaTeXe{}のネイティブモードで、上に示した変更のみで処理できます。
もう一度言いますが、あなたのドキュメントがネイティブモードで働くかどうかは、実際に試すことです。

%\section{Local modifications}
\section{自分用の変更}
\label{sec:loc}

\NEWdescription{1995/12/01}
%There are two common types of local modifications that can be done
%very simply.  Do not forget that documents produced using such
%modifications will not be usable at other places (such documents are
%called `non-portable').
非常に簡単に行うことができる一般的な修正方法が2つあります。
ただしそのような変更を使用して作成したドキュメントは、他の場所では使用できないことを忘れないでください(このようなドキュメントは`可搬性がない'と呼ばれます)。

%One type of modification is the use of personal commands for commonly
%used symbols or constructions.  These should be put into a package
%file (for example, one called \texttt{mymacros.sty}) and loaded by
%putting |\usepackage{mymacros}| in the document preamble.
修正の1つのタイプは、頻繁に使用する記号または構造に対して自分用のコマンドの使うことです。
これらをパッケージファイル(たとえば、\texttt{mymacros.sty}と呼ばれるファイル)に入れて、ドキュメントのプリアンブルで|\usepackage{mymacros}|として読み込みます。

%Another type is a local document class that is very similar to one of
%the standard classes but contains some straightforward modifications
%such as extra environments, different values for some parameters, etc.
%These should be put into a class file; here we shall describe a simple
%method of constructing such a file using, as an example, a class
%called \textsf{larticle} that is very similar to the \textsf{article}
%class.
もう一つの方法は、標準クラスに似せて自分用のクラスファイルを作ることです。
それには、環境の追加、いくつかのパラメータの値を変更するなど、直接的な変更も含みます。
これらはクラスファイルに入れる必要があります。
ここでは、\textsf {article}クラスと非常によく似ている\textsf {larticle}というクラスを使用して、そのようなファイルを構築する簡単な方法を説明します。

\NEWfeature{1995/12/01}
%The class file called \texttt{larticle.cls} should (after the
%preliminary identification commands) start as follows:
クラスファイル\texttt{larticle.cls}は(予備的な識別コマンドの後で)、つぎのように始めなければなりません。
\begin{verbatim}
   \LoadClassWithOptions{article}
\end{verbatim}
%This command should be followed by whatever additions and changes you
%wish to make to the results of reading in the file
%\texttt{article.sty}.
このコマンドに続くのは、\texttt{article.sty}ファイルを読み込んだ後で行いたい追加や変更です。

%The effect of using the above |\LoadClassWithOptions| command is to
%load the standard class file \textsf{article} with whatever options
%are asked for by the document.  Thus a document using your
%\textsf{larticle} class can specify any option that could be specified
%when using the standard \textsf{article} class; for example:
上記の|\LoadClassWithOptions|コマンドを使った効果は、標準クラスファイル\textsf{article}をドキュメントで指定されたオプションで読み込むことです。したがって、\textsf{larticle}クラスを使用するドキュメントでも標準の\textsf{article}クラスを使用するときに指定できるオプションを指定できます。
\begin{verbatim}
   \documentclass[a4paper,twocolumn,dvips]{larticle}
\end{verbatim}


%\section{Problems}
\section{いくつかの問題}
\label{Sec:problems}

%This section describes some of the things which may go wrong when
%using \LaTeXe, and what you can do about it.
この説では\LaTeXe{}の使い方を間違った時に、どうすればよいかを説明します。

%\subsection{New error messages}
\subsection{新しいエラーメッセージ}

%\LaTeXe{} has a number of new error messages.
%Please also note that many error messages now produce further helpful
%information if you press |h| in response to the error prompt.
\LaTeXe{}には新しいエラーメッセージがたくさんあります。
これらのエラーメッセージは、エラープロンプトに対して|h|を押せば、多くの助けとなる情報が得れれます。
\begin{decl}
|Option clash for package |\m{package}|.|
\end{decl}
%The named package has been loaded twice with different options.  If
%you enter |h| you will be told what the options were, for example, if
%your document contained:
パッケージは、異なるオプションで2回呼び出されました。
|h|を押して、どのオプションを指定したかったのかを与えます。
たとえば問題のドキュケントが
\begin{verbatim}
   \usepackage[foo]{fred}
   \usepackage[baz]{fred}
\end{verbatim}
%then you will get the error message:
であるとすれば、エラーメッセージは
\begin{verbatim}
   Option clash for package fred.
\end{verbatim}
%and typing |h| at the |?| prompt will give you:
となるでしょう。ここで|?|プロンプトに対して|h|とタイプすると
\begin{verbatim}
   The package fred has already been loaded with options:
     [foo]
   There has now been an attempt to load it with options:
     [baz]
   Adding the line:
     \usepackage[foo,baz]{fred}
   to your document may fix this.
   Try typing <<return>> to proceed.
\end{verbatim}
%The cure is, as suggested, to load the package with both sets of
%options.  Note that since \LaTeX{} packages can call other packages,
%it is possible to get a package option clash without explicitly
%requesting the same package twice.
と応答します。
訂正するためには、そこで示されている通り、異なるオプションを読み込みます。
\LaTeX{}パッケージは、他のパッケージを呼び出すこともできるので、
明示的に同じパッケージを2回呼び出さなくてもオプションの衝突が起こることがあります。

\begin{decl}
   |Command |\m{command}| not provided in base NFSS.|
\end{decl}
%The \m{command} is not provided by default in \LaTeXe.  This error is
%generated by using one of the commands:
LaTeXe{}ではデフォルトでは\m{command}コマンドは与えられていません。
このエラーは、次のコマンドの一つを呼び出すと発生します。
\begin{verbatim}
   \mho \Join \Box \Diamond \leadsto
   \sqsubset \sqsupset \lhd \unlhd \rhd \unrhd
\end{verbatim}
%which are now part of the \textsf{latexsym} package.
%The cure is to add:
これらは現在では\textsf{latexsym}パッケージの一部です。
解決策は、ドキュメントのプリアンブルで、次のようにして加えることです。
\begin{verbatim}
   \usepackage{latexsym}
\end{verbatim}
%in the preamble of your document.

\begin{decl}
   |LaTeX2e command <command> in LaTeX 2.09 document.|
\end{decl}
%The \m{command} is a \LaTeXe{} command but this is a \LaTeX~2.09
%document.  The cure is to replace the command by a \LaTeX~2.09
%command, or to run document in native mode, as described in
%Section~\ref{Sec:native}.
\m{command}は\LaTeXe{}コマンドの一つですが、これは\LaTeX~2.09ドキュメントです。
解決策は、コマンドを\LaTeX~2.09のコマンドに置き換えることか、
あるいは、\ref{Sec:native}節で説明したように
ネイテイィブモードでドキュメントを処理することです。

\begin{decl}
   |NFSS release 1 command \newmathalphabet found.|
\end{decl}
%The command |\newmathalphabet| was used by the New Font Selection
%Scheme Release 1 but it has now been replaced by
%|\DeclareMathAlphabet|, the use of which is described in
%\emph{\fntguide}.
コマンド|\newmathalphabet|が新しいフォント選択スキームのリリース1で使われてましたが、
これは|\DeclareMathAlphabet|で置き換えられました。使い方は\emph{\fntguide}を参照してください。

%The best cure is to update the package which contained the
%|\newmathalphabet| command.  Find out if there is a new release of
%the package, or (if you wrote the package yourself) consult
%\emph{\fntguide} for the new syntax of font commands.
一番よい解決策は、|\newmathalphabet|コマンドを含んでいるパッケージを更新することです。
新しいパッケージを探すか、(あなたが作者ならば)\emph{\fntguide} を参照して新しいコマンドを知ることです。

%If there is no updated version of the package then you can cure this
%error by using the \textsf{newlfont} or \textsf{oldlfont} package,
%which tells \LaTeX{} which version of |\newmathalphabet| should be
%emulated.
このパッケージの更新さればバージョンが見つからない時は、
\textsf{newlfont}あるいは\textsf{oldlfont}パッケージを使って解決できます。
これは\LaTeX{}に対して、|\newmathalphabet|のどのバージョンをエミュレートするかを伝えます。

%You should use \textsf{oldlfont} if the document selects math fonts with
%syntax such as this:
つぎのような構文
\begin{quote}
  |{\cal A}|, etc.
\end{quote}
でドキュメントの数学フォントで選択しているのなら\textsf{oldlfont}を使うべきです。
%Use \textsf{newlfont} if the document's syntax is like this:
ドキュメントの構文が
\begin{quote}
  |\cal{A}|, etc.
\end{quote}
であるならば、\textsf{newlfont}を使います。

\begin{decl}
   |Text for \verb command ended by end of line.|
\end{decl}
%The |\verb| command has been begun but not ended on that line.  This
%usually means that you have forgotten to put in the end-character of
%the |\verb| command.
|\verb|コマンドが始まっていますが、この行の終わっていません。
これは|\verb|コマンドの末端文字を忘れていているのです。

\begin{decl}
   |Illegal use of \verb command.|
\end{decl}
%The |\verb| command has been used inside the argument of another
%command.  This has never been allowed in \LaTeX{}---often producing
%incorrect output without any warning---and so \LaTeXe{} produces
%an error message.
|\verb|コマンドは別のコマンドの引数で使われています。
これは\LaTeX{}では許されていません。警告なく間違った出力を生成します。
\LaTeXe{}はエラーメッセージを出します。

%\subsection{Old internal commands}
\subsection{古い内部コマンド}
\label{Sec:oldinternals}

%A number of \LaTeX~2.09 internal commands have been removed, since
%their functionality is now provided in a different way.  See
%\emph{\clsguide} for more details of the new, supported interface for
%class and package writers.
\LaTeX~2.09の多くの内部コマンドは削除されましたので、それらの機能は異なる方法で行われています。
新しいことやクラスとパッケージ作成者のためのサポートされているインタフェースについては\emph{\clsguide}に詳しく述べてあります。

\begin{decl}
   |\tenrm| |\elvrm| |\twlrm| \dots\\
   |\tenbf| |\elvbf| |\twlbf| \dots\\
   |\tensf| |\elvsf| |\twlsf| \dots\\
   $\vdots$
\end{decl}
%These commands provided access to the seventy fonts preloaded by
%\LaTeX~2.09.  In contrast, \LaTeXe{} normally preloads at most
%fourteen fonts, which saves a lot of font memory; but a consequence is
%that any \LaTeX{} file which used the above commands to directly
%access fonts will no longer work.
これらのコマンドは\LaTeX~2.09でプレロードされる70個のフォントへのアクセスを提供していていました。
対照的に\LaTeXe{}では、14個のフォントをプレロードして、メモリを節約します。
この結果、上のコマンドを使ってフォントに直接アクセする\LaTeX{}ファイルは、もはや機能しません。

%Their use will usually produce an error message such as:
通常、次のようなエラーエッセージが1つ使われす。
\begin{verbatim}
   ! Undefined control sequence.
   l.5 \tenrm
\end{verbatim}
%The cure for this is to update the document to use the new
%font-changing commands provided by \LaTeXe; these are described in
%\emph{\fntguide}.
解決策は、\LaTeXe{}で使われるフォント変更コマンドを使うようにドキュメント更新することです。
詳しいことは\emph{\fntguide}に説明されています。

%If this is not possible then, as a last resort, you can
%use the \textsf{rawfonts} package, which loads the
%seventy \LaTeX~2.09 fonts and provides direct access to them using the
%old commands.  This takes both time and memory.  If you do not
%wish to load all seventy fonts, you can select some of them by using the
%|only| option to \textsf{rawfonts}.  For example, to load only |tenrm|
%and |tenbf| you write:
これが不可能な場合は、最後の手段として、\textsf{rawfonts}パッケージを使うことができます。
これは70個の\LaTeX~2.09フォントを読み込み、それらに古いコマンドで直接アクセスします。
これには時間とメモリの両方が必要となります。
70個のフォントすべてをロードしたくない場合は、\textsf {rawfonts}を|only|オプションを使用すれば、必要なフォントを選択できます。
たとえば、|tenrm|と|tenbf|が必要なら、
\begin{verbatim}
   \usepackage[only,tenrm,tenbf]{rawfonts}
\end{verbatim}
と書きます。

%The \textsf{rawfonts} package is distributed as part of the \LaTeX{}
%tools software, see Section~\ref{Sec:st-pack}.
\textsf{rawfonts}パッケージはツールソフトウェアとして\LaTeX{}の一部として配布されています。
\ref{Sec:st-pack}節を参照してください。

%\subsection{Old files}
\subsection{古いファイル}

%One of the more common mistakes in running \LaTeX{} is to read in old
%versions of packages instead of the new versions.  If you get an
%incomprehensible error message from a standard package, make sure you
%are loading the most recent version of the package.  You can find out
%which version of the package has been loaded by looking in the log
%file for a line like:
\LaTeX{}を実行する上でよくある間違いの1つは、新しいバージョンではなく古いバージョンのパッケージを読み込んでしまうことです。
標準パッケージから理解できないエラーメッセージが表示された場合は、パッケージの最新バージョンをロードしていることかどうかを確認してください。
ロードされたパッケージのバージョンは、ログファイルに次のようなに表示されるので、確認できます。
\begin{verbatim}
   Package: fred 1994/06/01 v0.01 Fred's package.
\end{verbatim}
%You can use the \m{release-date} options to |\documentclass| and
%|\usepackage| to make sure that you are getting a suitably recent copy
%of the document class or package.  This is useful when sending a
%document to another site, which may have out-of-date software.
適切な最新のドキュメントクラスあるいはパッケージを使ったかどうかは|\documentclass|と|\usepackage|に\m{release-date}オプションを使うことではっきりできます。
この方法は、古いソフトウェアを使っているかもしれない他の人や組織にドキュメントを送る時に便利です。

%\subsection{Where to go for more help}
\subsection{さらに助けが必要な場合}

%If you can't find the answer for your problem here, try looking in
%\emph{\LaTeXbook} or \emph{\LaTeXcomp}.  If you have a problem with
%installing \LaTeX, look in the installation guide files which come with
%the distribution.
遭遇した問題の解決策が見つからない場合は、\emph{\LaTeXbook}あるいは\emph{\LaTeXcomp}の中を探してください。
\LaTeX{}のインストールに問題があるのならば、配布ファイルに付属するインストールガイドをみてください。

%If this doesn't help, contact your local \LaTeX{} guru or local
%\LaTeX{} mailing list.
それでもうまくいかない時、身近なところにいる\LaTeX{}の専門家や\LaTeX{}のメーリングリストにあたってください。

%If you think you've discovered a bug then please report it!  First,
%you should find out if the problem is with a third-party package or
%class.  If the problem is caused by a package or class other than
%those listed in Section~\ref{Sec:class+packages} then please report
%the problem to the author of the package or class, not to the \LaTeX3
%project team.
あなたがバグを発見したと思うなら、それを報告してください!
まず最初に、問題が第三者のパッケージまたはクラスであるかどうかを調べます。
\ref{Sec:class+packages}節に記載されていないパッケージやクラスに引き起こされた問題であれば、\LaTeX3{}プロジェクトチーム宛ではなく、そのパッケージまたはクラスの作成者に報告してください。

%If the bug really is with core \LaTeX{} then you should create a
%\emph{short}, \emph{self-contained} document which exhibits the
%problem.  You should run a \emph{recent} (less than a year old)
%version of \LaTeX{} on the file and then run \LaTeX{} on
%|latexbug.tex|.  This will create an error report which you should
%send, together with the sample document and log file, to the
%\LaTeX{} bugs address which can be found in the file
%|latexbug.tex| or |bugs.txt|.
バグが本当に\LaTeX{}の中にあるのならば、問題を明確に示す\emph{短く}て\emph{自己完結した}ドキュメントを作成してください。
そのファイルを\emph{最新の}(1年以内の)バージョンの\LaTeX{}で実行し、そして|latexbug.tex|を\LaTeX{}で実行する必要があります。
こうすると、|latexbug.tex|あるいは|bugs.txt|に記載されている\LaTeX{}バグ用のアドレスに送るべきエラーリポートが、サンプルドキュメントとログファイルとともにできます。

%\section{Enjoy!}
\section{楽しんで!}
\label{Sec:enjoy}

%We certainly hope you will enjoy using the new standard \LaTeX{} but,
%if this is not possible, we hope that you will enjoy success and
%fulfillment as a result of the documents which it will help you to
%create.
新しい企画の\LaTeX{}楽しんでくれることを望みますが、これが不可能な場合は、このドキュメントが助けになり、うまくいくことを望みます。

%If you find that the contribution of \LaTeX{} to your life is such
%that you would like to support the work of the project team, then
%please read Section~\ref{Sec:ltx3} and discover practical ways to do
%this.
\LaTeX{}が、あなたの役に立ち、プロジェクトチームの仕事をサポートしたくなったら、\ref{Sec:ltx3}節を読んで実践的な方法を見つけてください。

\begin{thebibliography}{1}

\bibitem{A-W:GRM97}
Michel Goossens, Sebastian Rahtz and Frank Mittelbach.
\newblock {\em The {\LaTeX} Graphics Companion}.
\newblock Addison-Wesley, Reading, Massachusetts, 1997.


\bibitem{A-W:GR99}
Michel Goossens and Sebastian Rahtz.
\newblock {\em The {\LaTeX} Web Companion}.
\newblock Addison-Wesley, Reading, Massachusetts, 1999.


\bibitem{A-W:DEK91}
Donald~E. Knuth.
\newblock {\em The \TeX book}.
\newblock Addison-Wesley, Reading, Massachusetts, 1986.
\newblock Revised to cover \TeX3, 1991.


\bibitem{A-W:LLa94}
Leslie Lamport.
\newblock {\em {\LaTeX:} A Document Preparation System}.
\newblock Addison-Wesley, Reading, Massachusetts, second edition, 1994.

\bibitem{A-W:MG2004}
Frank Mittelbach and Michel Goossens.
\newblock {\em The {\LaTeX} Companion second edition}.
\newblock With Johannes Braams, David Carlisle, and Chris Rowley.
\newblock Addison-Wesley, Reading, Massachusetts, 2004.


\end{thebibliography}

\end{document}
